%%%%%%%%%%%%%%%%%%%%%%%%%%%%%%%%%%%%%%%%%%%%%%%%%%%%%%%%%%%%%%%%%%%%%%%%%%%%%%%
% A clean template for an academic CV. This is a short summary version.
%
% Uses tabularx to create two column entries (date and job/edu/citation).
% Defines commands to make adding entries simpler.
%
%%%%%%%%%%%%%%%%%%%%%%%%%%%%%%%%%%%%%%%%%%%%%%%%%%%%%%%%%%%%%%%%%%%%%%%%%%%%%%%

\documentclass[9pt,a4paper]{article}

% Useful aliases
\newcommand{\FNUniv}{First Nations University of Canada}
\newcommand{\UofR}{University of Regina}


% Identifying information
\newcommand{\Title}{Curriculum Vit\ae\ Summary}
\newcommand{\FirstName}{Edward}
\newcommand{\MiddleName}{Jon}
\newcommand{\LastName}{Doolittle}
\newcommand{\Initials}{E}
\newcommand{\MyName}{\FirstName\ \MiddleName\ \LastName}
\newcommand{\Me}{\underline{\LastName, \Initials}}  % For citations
\newcommand{\Email}{edoolittle@firstnationsuniversity.ca}
\newcommand{\PersonalWebsite}{www.fnuniv.ca/academic/faculty/dr-edward-doolittle}
\newcommand{\LabWebsite}{www.fnuniv.ca}
\newcommand{\ORCID}{0009-0000-2155-8749}
\newcommand{\GitHubProfile}{edoolittle}

% Names for citing coauthors
\newcommand{\Shalene}{Jobin, S}
\newcommand{\Shaun}{Fallat, S}
\newcommand{\Layne}{Burns, L}
\newcommand{\Cynthia}{Nicol, C}
\newcommand{\Florence}{Glanfield, F}
\newcommand{\Jennifer}{Thom, J}
\newcommand{\Malabika}{Pramanik, M}
\newcommand{\Laleh}{Behjat, L}
\newcommand{\Marc}{Frappier, M}
\newcommand{\Viqar}{Husain, V}
\newcommand{\Mark}{Lewis, M}
\newcommand{\Lei}{Sun, L}
\newcommand{\Betty}{McKenna, B}
\newcommand{\Dwayne}{Donald, D}
\newcommand{\Gladys}{Sterenberg G}


% Load packages
%%%%%%%%%%%%%%%%%%%%%%%%%%%%%%%%%%%%%%%%%%%%%%%%%%%%%%%%%%%%%%%%%%%%%%%%%%%%%%%

% Full Unicode support for non-ASCII characters
\usepackage[utf8]{inputenc}
\usepackage[english]{babel}
\usepackage[TU]{fontenc}

% Set main fonts
\usepackage[sfdefault]{atkinson}
\usepackage[ttdefault]{sourcecodepro}

% Icon fonts
\usepackage{fontawesome5}
\usepackage{academicons}

% Disable hyphenation
\usepackage[none]{hyphenat}

% Control the font size
\usepackage{anyfontsize}

% For fancy and multipage tables
\usepackage{tabularx}
\usepackage{ltablex}

% For new environments
\usepackage{environ}

% Manage dates and times
\usepackage{datetime}

% Set the page margins
\usepackage{geometry}

% To get the total page numbers (\pageref{LastPage})
\usepackage{lastpage}

% Control spacing in enumerates
\usepackage{enumitem}

% Use custom colors
\usepackage[usenames,dvipsnames]{xcolor}

% Configure section titles
\usepackage[nobottomtitles]{titlesec}
\renewcommand{\bottomtitlespace}{.1\textheight}

% Fancy header configuration
\usepackage{fancyhdr}

% Control PDF metadata and links
\usepackage[colorlinks=true]{hyperref}

% This should break long URLs per
% https://tex.stackexchange.com/questions/386495/how-to-wrap-a-url-or-reference
% but does not seem to help
%% Wrap long links
%\usepackage[ocgcolorlinks]{ocgx2}

% Template configuration
%%%%%%%%%%%%%%%%%%%%%%%%%%%%%%%%%%%%%%%%%%%%%%%%%%%%%%%%%%%%%%%%%%%%%%%%%%%%%%%

\geometry{%
  margin=12.5mm,
  headsep=1mm,
  headheight=0mm,
  footskip=5mm,
  includehead=true,
  includefoot=true
}

% Custom colors
\definecolor{mediumgray}{gray}{0.5}
\definecolor{lightgray}{gray}{0.9}
\definecolor{mediumblue}{HTML}{2060c2}
\definecolor{lightblue}{HTML}{a0c3ff}
%\definecolor{darkblue}{HTML}{16181d}
%\definecolor{darkblue}{HTML}{000080}
%\definecolor{darkblue}{HTML}{000060}
%\definecolor{darkblue}{HTML}{000040}
\definecolor{darkblue}{gray}{0.0} % links identified by mouseover, not colour

% No indentation
\setlength\parindent{0cm}

% Increase the line spacing
\renewcommand{\baselinestretch}{1.1}
% and the spacing between rows in tables
\renewcommand{\arraystretch}{1.25}

% Remove space between items in itemize and enumerate
\setlist{nosep}

% Set the spacing and format of sections
\titleformat{\section}
  {\normalfont\Large\mdseries} % format
  {} % label
  {0pt} % separation (left separation for hang)
  {} % text before title
  [\titlerule] % text after title
\titlespacing*{\section}
  {0pt} % left pad
  {0.1cm} % before
  {0cm} % after

% Disable number of sections. Use this instead of "section*" so that
% the sections still appear as PDF bookmarks. Otherwise, would have
% to add the table of contents entries manually.
\makeatletter
\renewcommand{\@seccntformat}[1]{}
\makeatother

% Define a new environment to place all CV entries in a 2-column table.
% Left column are the dates, right column the entries.
\newcommand{\TablePad}{\vspace{-0.2cm}}
\NewEnviron{EntriesTableDuration}{
\TablePad
\begin{tabularx}{\textwidth}{@{}p{0.10\textwidth}@{\hspace{0.02\textwidth}}p{0.88\textwidth}@{}}
  \BODY
\end{tabularx}
\TablePad
}
\NewEnviron{EntriesTableYear}{
\TablePad
\begin{tabularx}{\textwidth}{@{}p{0.05\textwidth}@{\hspace{0.01\textwidth}}p{0.94\textwidth}@{}}
  \BODY
\end{tabularx}
\TablePad
}

% Macros to set the year and duration on the left column
\newcommand{\Duration}[2]{\fontsize{10pt}{0}\selectfont \texttt{#1-#2}}
\newcommand{\Year}[1]{\fontsize{10pt}{0}\selectfont \texttt{#1}}
\newcommand{\Ongoing}{on}
\newcommand{\Future}{future}

% Macros to add links and mark publications
\newcommand{\DOI}[1]{doi:\href{https://doi.org/#1}{#1}}
\newcommand{\Website}[1]{\href{https://#1}{#1}}
\newcommand{\Preprint}[1]{Preprint: \href{https://doi.org/#1}{#1}}
\newcommand{\GitHub}[1]{\faGithub{} \href{https://github.com/#1}{#1}}
\newcommand{\Data}[1]{\faChartBar{} doi:\href{https://doi.org/#1}{#1}}
\newcommand{\MYhref}[3][darkblue]{\href{#2}{\color{#1}{#3}}}

% Define command to insert month name and year as date
\newdateformat{monthyear}{\monthname[\THEMONTH], \THEYEAR}

% Configure a fancy footer
\newcommand{\Separator}{\hspace{3pt}|\hspace{3pt}}
\newcommand{\FooterFont}{\footnotesize\color{mediumgray}}
\pagestyle{fancy}
\fancyhf{}
\lfoot{%
  \FooterFont{}
  \MyName{}
  \Separator{}
  \Title{}
}
\rfoot{%
  \FooterFont{}
  Last updated: \monthyear\today{}
  \Separator{}
  \thepage\space of\space \pageref*{LastPage}
}
\renewcommand{\headrulewidth}{0pt}
\renewcommand{\footrulewidth}{1pt}
\preto{\footrule}{\color{lightgray}}

% Metadata for the PDF output and control of hyperlinks
\hypersetup{
  colorlinks,
  allcolors=mediumblue,
  breaklinks=true,
  pdftitle={\Title{} - \MyName},
  pdfauthor={\MyName},
}
% This should break long URLs per
% https://tex.stackexchange.com/questions/386495/how-to-wrap-a-url-or-reference
% but does not seem to help
%\makeatletter
%    \g@addto@macro{\UrlBreaks}{\do\/\do\-\do\_}
%\makeatother

%%%%%%%%%%%%%%%%%%%%%%%%%%%%%%%%%%%%%%%%%%%%%%%%%%%%%%%%%%%%%%%%%%%%%%%%%%%%%%%
\begin{document}

\begin{minipage}[t]{0.5\textwidth}
  {\fontsize{20pt}{0}\selectfont\MyName}
\end{minipage}
\begin{minipage}[t]{0.5\textwidth}
  \begin{flushright}
    \Title{}
  \end{flushright}
\end{minipage}
\\[-0.1cm]
\textcolor{lightgray}{\rule{\textwidth}{3pt}}
\begin{minipage}[t]{0.45\textwidth}
  ORCID: \href{https://orcid.org/\ORCID}{\ORCID}
  \\
  Email: \href{mailto:\Email}{\Email}
  \\
  Website: \Website{\PersonalWebsite}
  %\\
  %Research Group: \Website{\LabWebsite}
  \\
  Latest CV: \Website{github.com/edoolittle/cv/raw/pdf/cv.pdf}
\end{minipage}
\begin{minipage}[t]{0.55\textwidth}
  \begin{flushright}
  Associate Professor of Mathematics
  \\
  Departament of Indigenous Knowledge and Science
  \\
  First Nations University of Canada
  \\
  1 First Nations Way, Regina, Saskatchewan, S4S 7K2, Canada
  \end{flushright}
\end{minipage}
\vspace{0.3cm}

%%%%%%%%%%%%%%%%%%%%%%%%%%%%%%%%%%%%%%%%%%%%%%%%%%%%%%%%%%%%%%%%%%%%%%%%%%%%%%%
\section{Current Professional Appointments}

\begin{EntriesTableDuration}
  \Duration{2024}{\Ongoing} & \textbf{Associate Dean, Research and
    Graduate Programs}, \MYhref{https://www.fnuniv.ca}{\FNUniv}\
  \Website{www.fnuniv.ca} %
  \\ %
  \Duration{2024}{\Ongoing} & \textbf{Co-chair (Indigenous Research)},
  \MYhref{https://www.uregina.ca/research/office-research-services/human-research/index.html}{Research
    Ethics Board}, \MYhref{https://www.uregina.ca}{\UofR}\
  \Website{www.uregina.ca} %
  \\ %
  \Duration{2014}{\Ongoing} & \textbf{Associate Professor of
    Mathematics}, \MYhref{https://www.fnuniv.ca}{\FNUniv}\
  \Website{www.fnuniv.ca} %
\end{EntriesTableDuration}

%%%%%%%%%%%%%%%%%%%%%%%%%%%%%%%%%%%%%%%%%%%%%%%%%%%%%%%%%%%%%%%%%%%%%%%%%%%%%%%
\section{Previous Professional Appointments}

\begin{EntriesTableDuration}
  \Duration{2011}{2014} & \textbf{Associate Professor of Mathematics
    and Department Head}, \MYhref{https://www.fnuniv.ca}{\FNUniv}
  % \Website{www.fnuniv.ca}
  \\
  \Duration{2009}{2011} & \textbf{Associate Professor of Mathematics},
  \MYhref{https://www.fnuniv.ca}{\FNUniv}\ \Website{www.fnuniv.ca}
  \\
  \Duration{2008}{2009} & \textbf{Assistant Professor of Mathematics},
  \MYhref{https://www.fnuniv.ca}{\FNUniv}\ \Website{www.fnuniv.ca}
  \\
  \Duration{2005}{2008} & \textbf{Assistant Professor of Mathematics},
  \MYhref{https://www.uregina.ca}{\UofR}\ \Website{www.uregina.ca}
  \\
  \Duration{2002}{2005} & \textbf{Assistant Professor of Mathematics
    and Department Head}, \MYhref{https://www.fnuniv.ca}{\FNUniv}
  % \Website{www.fnuniv.ca}
  \\
  \Duration{2001}{2002} & \textbf{Assistant Professor of Mathematics},
  \MYhref{https://www.fnuniv.ca/about-us/}{Saskatchewan
    Indian Federated College}
\end{EntriesTableDuration}

%%%%%%%%%%%%%%%%%%%%%%%%%%%%%%%%%%%%%%%%%%%%%%%%%%%%%%%%%%%%%%%%%%%%%%%%%%%%%%%
\section{Leadership and Involvement}

\begin{EntriesTableDuration}
  \Duration{2025}{\Ongoing} & \textbf{Board Member},
  \MYhref{https://www.pims.math.ca/people/board}{Board of Directors},
  \MYhref{https://www.pims.math.ca/}{Pacific Institute for the
    Mathematical Sciences (PIMS)} %
  \newline %
  \Website{www.pims.math.ca/people/board} %
  \\ %
  \Duration{2023}{\Ongoing} & \textbf{Chair},
  \MYhref{https://cms.math.ca/about-the-cms/governance/committees/\#rmc}{Mathematics
    and Reconciliation Committee},
  \MYhref{https://cms.math.ca/}{Canadian Mathematical Society (CMS)} %
  \newline %
  \Website{cms.math.ca/about-the-cms/governance/committees/\#rmc} %
  \\ %
  \Duration{2022}{\Ongoing} & \textbf{Executive Member},
  \MYhref{https://carmamaths.org/}{Computer-Assisted Research
    Mathematics and its Applications (CARMA)}, Australia %
  \Website{carmamaths.org/people/}
  \\ %
  \Duration{2022}{\Ongoing} & \textbf{Committee Member},
  \MYhref{https://www.fields.utoronto.ca/sites/default/files/Fields
    Institute Annual Report 2021-22.pdf}{Equity, Diversity and
    Inclusion Advisory Committee},
  \MYhref{http://www.fields.utoronto.ca/}{Fields Institute for
    Research in Mathematical Sciences (Fields)}
  \Website{www.fields.utoronto.ca} %
  \\ %
  \Duration{2021}{\Ongoing} & \textbf{Committee Member},
  \MYhref{https://www.pims.math.ca/people/committees/indigenous-engagement-committee}{Indigenous
    Engagement Committee}, \MYhref{https://www.pims.math.ca/}{Pacific
    Institute for the Mathematical Sciences (PIMS)}
  \Website{www.pims.math.ca/people/committees/indigenous-engagement-committee} %
  \\ %
  \Duration{2020}{\Ongoing} & \textbf{Board Member},
  \MYhref{https://www.birs.ca/about/governance/scientific-management/Equity-Diversity-Inclusion-Board}{Equity,
    Diversity and Inclusion Advisory Board},
  \MYhref{https://www.birs.ca}{Banff International Research Station (BIRS)}
  \Website{www.birs.ca/about/governance/scientific-management/Equity-Diversity-Inclusion-Board/current-members-EDIB} %
  \\ %
  \Duration{2019}{\Ongoing} & \textbf{Judge}, National Judging Team,
  \MYhref{https://youthscience.ca}{Youth Science Canada (YSC)}
  \Website{youthscience.ca} %
  \\ %
  \Duration{2018}{\Ongoing} & \textbf{Trustee},
  \MYhref{https://www.urfa.ca/committees/trust-fund}{Trust Fund
    Committee}, \MYhref{https://www.urfa.ca}{University of Regina
    Faculty Association} %
  %\newline %
  \Website{urfa.ca/committees/trust-fund} %
\end{EntriesTableDuration}

%%%%%%%%%%%%%%%%%%%%%%%%%%%%%%%%%%%%%%%%%%%%%%%%%%%%%%%%%%%%%%%%%%%%%%%%%%%%%%% 
\section{Education}

\begin{EntriesTableDuration}
  \Duration{1992}{1997} & \textbf{PhD in Mathematics},
  \MYhref{https://www.utoronto.ca}{University of Toronto}
  \Website{www.utoronto.ca} %
  %\newline %
  %\textbf{Thesis:} A Parametrix for Stable Step Two Hypoelliptic
  %Partial Differential Operators \textbf{Advisor:}
  %\MYhref{https://www.mathgenealogy.org/id.php?id=15110}{Peter
  %  Greiner} %
  \newline %
  \Website{utoronto.scholaris.ca/server/api/core/bitstreams/a6c7bee0-0bce-49dd-8044-e0695623a0dc/content}
  \\ %
  \Duration{1990}{1992} & \textbf{MSc in Mathematics},
  \MYhref{https://www.utoronto.ca}{University of Toronto},
  \Website{www.utoronto.ca} %
  \\ %
  \Duration{1985}{1990} & \textbf{BSc in Mathematics},
  \MYhref{https://www.utoronto.ca}{University of Toronto},
  \Website{www.utoronto.ca} %
\end{EntriesTableDuration}

%%%%%%%%%%%%%%%%%%%%%%%%%%%%%%%%%%%%%%%%%%%%%%%%%%%%%%%%%%%%%%%%%%%%%%%%%%%%%%% 
\section{Indigenous Identity}

\begin{EntriesTableDuration}
  & I am a \textbf{Status Indian}, member of the \textbf{Lower Mohawk}
  band of \textbf{Six Nations}.
\end{EntriesTableDuration}

%%%%%%%%%%%%%%%%%%%%%%%%%%%%%%%%%%%%%%%%%%%%%%%%%%%%%%%%%%%%%%%%%%%%%%%%%%%%%%%
\section{Fellowships and Awards}

\begin{EntriesTableYear}
  \Year{2024} &
  \MYhref{https://cms.math.ca/awards/fellows-of-the-cms/}{\textbf{Fellow}}
  of the \MYhref{https://cms.math.ca}{Canadian Mathematical Society} %
  \newline %
  \Website{cms.math.ca/news-item/canadian-mathematical-societys-2024-class-of-fellows-announced/} %
  \\ %
  \Year{2023} &
  \MYhref{cms.math.ca/awards/adrien-pouliot-award/}{\textbf{Adrien
      Pouliot Award}}, \MYhref{https://cms.math.ca}{Canadian
    Mathematical Society} %
  \newline %
  \Website{cms.math.ca/news-item/dr-edward-doolittle-named-the-2023-adrien-pouliot-award-recipient/}
  %\Website{cms.math.ca/awards/adrien-pouliot-award/} %
  \\ %
  \Year{1992} &
  \MYhref{https://www.gg.ca/en/honours/governor-generals-awards/governor-generals-academic-medal}{\textbf{Governor
      General's Gold Medal}}
  \Website{www.gg.ca/en/honours/recipients/116-5733} %
  %\\ %
  %\Year{1987} & \textbf{Honorable Mention},
  %\MYhref{https://maa.org/putnam/}{William Lowell Putnam Mathematical
  %  Competition} %
  %\newline %
  %\Website{kskedlaya.org/putnam-archive/putnam1987results.html} %
\end{EntriesTableYear}

%%%%%%%%%%%%%%%%%%%%%%%%%%%%%%%%%%%%%%%%%%%%%%%%%%%%%%%%%%%%%%%%%%%%%%%%%%%%%%% 
\section{Grants}

\begin{EntriesTableDuration}
  \Year{2026} &
  \MYhref{https://www.birs.ca/event/26w5629}{\textbf{Mathematics
      Bundle}}.  \MYhref{https://www.birs.ca}{Banff International
    Research Station (BIRS)} workshop
  \MYhref{https://www.birs.ca/event/26w5629}{26w5629}.  %
  \newline %
  \Florence{} (PI), \Me{}, \Betty{}
  \Website{www.birs.ca/event/26w5629} %
  \\ %
  \Duration{2025}{2027} &
  \MYhref{https://www.humanitiesresearch.org/profile-dr-edward-doolittle/}{\textbf{Word
      Puzzles in Indigenous Languages}}.
  \MYhref{https://www.humanitiesresearch.org}{Humanities Research
    Institute}
  \MYhref{https://www.humanitiesresearch.org/about/research-fellows/}{Fellowship},
  \MYhref{https://www.uregina.ca}{University of Regina}.  %
  \newline %
  \Me{} (PI). (\textbf{\$5,000 over 2 years})
  \Website{www.humanitiesresearch.org/profile-dr-edward-doolittle/} %
  \\ %
  \Year{2025} &
  \MYhref{https://www.birs.ca/events/2025/5-day-workshops/25w5472}{\textbf{The
      Math Bundle}}.  \MYhref{https://www.birs.ca}{Banff International
    Research Station (BIRS)} workshop
  \MYhref{https://www.birs.ca/events/2025/5-day-workshops/25w5472}{25w5472}.  %
  \newline %
  \Me{} (PI), \Florence{}, \Betty{}
  \Website{www.birs.ca/events/2025/5-day-workshops/25w5472} %
  \\ %
  \Duration{2024}{2030} &
  \MYhref{https://www.prairierelationality.ca/initiatives/cair}{\textbf{Critical
      Approaches to Indigenous Relationality}}.
  \MYhref{https://sshrc-crsh.canada.ca/en.aspx}{Social Sciences and
    Humanities Research Council (SSHRC)}
  \MYhref{https://sshrc-crsh.canada.ca/en/funding/opportunities/partnership-grants.aspx}{Partnership
    Grant}.
  \Website{www.prairierelationality.ca/cair-meet-our-team-co-applicants} %
  \newline %
  \Shalene{} (PI), et.~al, \Me{}, et.~al (\textbf{\$2.5 million over 6
    years}) %
  % \Website{www.sshrc-crsh.gc.ca/results-resultats/recipients-recipiendaires/2023/pg-sp-eng.aspx}
  \\ %
  \Duration{2024}{2029} &
  \MYhref{https://webapps.cihr-irsc.gc.ca/decisions/p/project\_details.html?applId=507374\&lang=en}{\textbf{SK-NEIHR:
      Indigenous Futurisms in Indigenous Health and Wellbeing: The
      natawihowin and ma\-ma\-wii\-ki\-ka\-yaahk Research, Training
      and Mentorship Networks}}.
  \MYhref{https://cihr-irsc.gc.ca/e/51161.html}{Network Environments
    for Indigenous Health Research (NEIHR)} operating grant,
  \MYhref{https://cihr-irsc.gc.ca/e/8172.html}{Institute of Indigenous
    Peoples' Health (IIPH)},
  \MYhref{https://cihr-irsc.gc.ca/e/193.html}{Canadian Institutes of
    Health Research (CIHR)} %
  %\newline % 
  \Website{research-groups.usask.ca/sk-neihr/index.php} %
  \newline %
  Henry, R (NPI), Campbell, L (PI), \Me{} (PI), et.~al
  (\textbf{\$5.575 million over 5 years}) %
  \\ %
  \Duration{2024}{2026} & \textbf{The Mathematics of Indigenous
    Games}.  \MYhref{https://www.fnuniv.ca}{First Nations University}
  \MYhref{https://www.fnuniv.ca/academic/research-office/fnuniv-board-of-governors-research-awards/}{Board
    of Governors Research Award}. %
  \newline %
  \Me{} (PI), \Shaun{}. (\textbf{\$5000 over 2 years}) %
  \\ %
  \Duration{2023}{2028} & \textbf{Mathematics Education for STEM as
    Place}.  \MYhref{https://sshrc-crsh.canada.ca/en.aspx}{Social
    Sciences and Humanities Research Council (SSHRC)}
  \MYhref{https://sshrc-crsh.canada.ca/en/funding/opportunities/insight-grants.aspx}{Insight
    Grant}. %
  \Website{www.sshrc-crsh.gc.ca/results-resultats/recipients-recipiendaires/2021/ig-ss-eng.aspx} %
  \newline %
  \Cynthia{} (PI), \Me{}, \Florence{}, \Jennifer{} (\textbf{\$360,000
    over 5 years}) %
  \\ %
  \Duration{2022}{2027} & \MYhref{https://www.birs.ca}{\textbf{Banff
      International Research Station}}.
  \MYhref{https://www.nserc-crnsg.gc.ca}{Natural Sciences and
    Engineering Research Council (NSERC)}
  \MYhref{https://www.nserc-crsng.gc.ca/professors-professeurs/Grants-Subs/DIS-ADIR_eng.asp}{Discovery
    Institutes Support Grants}. %
  \Website{www.nserc-crsng.gc.ca/ase-oro/Details-Detailles\_eng.asp?id=767960} %
  \newline %
  \Malabika{} (PI), \Laleh{}, \Me{}, \Marc{}, \Viqar{}, \Mark{},
  \Lei{}.  (\textbf{\$5 million over 5 years}) %
  % \newline %
  % \newline %
  % \Website{www.nserc-crsng.gc.ca/ase-oro/Details-Detailles\_eng.asp?id=758456} %
  % \Website{www.nserc-crsng.gc.ca/NSERC-CRSNG/FundingDecisions-DecisionsFinancement/ResearchGrants-SubventionsDeRecherche/ResultsGSCDetail-ResultatsCSSDetails\_eng.asp?Year=2022\&GSC=1625}
  \\ %
  \Year{2013} &
  \MYhref{https://www.birs.ca/events/2013/5-day-workshops/13w5120}{\textbf{Understanding
      Relationships between Aboriginal Knowledge Systems, Wisdom
      Traditions, and Mathematics: Research Possibilities}}.
  \MYhref{https://www.birs.ca}{Banff International Research Station
  (BIRS)} workshop
  \MYhref{https://www.birs.ca/events/2013/5-day-workshops/13w5120}{13w5120}.
  \newline %
  \Me{}; \Florence{}
  \Website{www.birs.ca/events/2013/5-day-workshops/13w5120} %
  \\ %
  \Duration{2010}{2012} & \textbf{Creating a Research Network to
    Develop an Understanding of Relationships between Aboriginal
    Knowledge Systems, Wisdom Traditions, and Mathematics Education}. 
  %\newline %
  \MYhref{https://sshrc-crsh.canada.ca/en.aspx}{Social Sciences and
    Humanities Research Council (SSHRC)}
  \MYhref{https://sshrc-crsh.canada.ca/transparency-transparence/disclosure-divulgation/grants-subventions/2010/jan_2010.pdf}{Aboriginal
    Research Development Grant}.  %
  \newline %
  \Florence{} (PI), \Dwayne{}, \Me{}, \Gladys{} (\textbf{\$25,000 over 2
  years}) %
  \newline %
  \Website{www.outil.ost.uqam.ca/crsh/Detail.aspx?Cle=78132\&Langue=2} %
  \\ %
  \Year{2008} & \textbf{MATH 104/105 Development of Online Calculus}.  %
  %\newline %
  Technology Enhanced Learning grant,
  \MYhref{https://www.saskatchewan.ca/government/government-structure/ministries/advanced-education}{Ministry
    of Advanced Education},
  \MYhref{https://www.saskatchewan.ca/government}{Government of
    Saskatchewan}. %
  \newline %
  Herman, Allen (PI) \& \Me{}. (\textbf{\$30,000}) %
  % \\
  % \Duration{2005}{2010} & \textbf{Understanding the Dynamics of Risk and
  %   Protective Factors in Promoting Success in Science and Mathematics}
  % \newline
  % NSERC CRYSTAL Grant.
  % \newline
  % Robinson, G. (PI), et. al. \& \Me{} (Collaborator)
  % (\$1,000,000 over 5 years)
  % \Website{www.nserc-crsng.gc.ca/ase-oro/index\_eng.asp?new}
  \\ %
  \Duration{1991}{1995} &
  \MYhref{https://www.nserc-crsng.gc.ca/students-etudiants/pg-cs/bellandpostgrad-belletsuperieures_eng.asp}{\textbf{NSERC
      Postgraduate Scholarship}} (\textbf{\$56,000 over 4 years}) %
  \newline %
  \Website{www.nserc-crsng.gc.ca/ase-oro/index\_eng.asp?new} %
\end{EntriesTableDuration}

%%%%%%%%%%%%%%%%%%%%%%%%%%%%%%%%%%%%%%%%%%%%%%%%%%%%%%%%%%%%%%%%%%%%%%%%%%%%%%% 
\section{Supervision}

\begin{EntriesTableDuration}
  \Duration{2024}{\Ongoing} & Layne Burns, MSc in Mathematics,
  \MYhref{https://www.uregina.ca}{University of Regina}
  (\textbf{Co-supervisor}) %
  \\ %
  \Duration{2024}{\Ongoing} & Whitney Ogle, MA in Indigenous
  Education, \MYhref{https://www.uregina.ca}{University of Regina}
  (\textbf{Committee Member}) %
  \\ %
  \Duration{2023}{2024} & Layne Burns,
  \MYhref{https://www.nserc-crsng.gc.ca}{Natural Sciences and
    Engineering Research Council (NSERC)}
  \MYhref{https://www.nserc-crsng.gc.ca/students-etudiants/ug-pc/usra-brpc_eng.asp}{Undergraduate
    Student Research Award (USRA)} in Mathematics,
  \MYhref{https://www.fnuniv.ca}{First Nations University of Canada}
  (\textbf{Supervisor}) %
  \\ %
  \Duration{2023}{\Ongoing} & John Porrit, PhD in Mathematics
  Education, \MYhref{https://www.uregina.ca}{University of Regina}
  (\textbf{Co-supervisor}) %
  \\ %
  \Duration{2023}{\Ongoing} & Ehdaa Matia, PhD in Mathematics
  Education, \MYhref{https://www.uregina.ca}{University of Regina}
  (\textbf{Committee Member}) %
  \\ %
  \Duration{2022}{2023} & Shana Graham, Postdoctoral Fellowship,
  \MYhref{https://www.uregina.ca}{University of Regina}
  (\textbf{Co-supervisor}) %
  \\ %
  \Year{2022} & Myron Medina, PhD in Curriculum Studies,
  \MYhref{https://www.ubc.ca}{University of British Columbia}
  (\textbf{External Examiner}) %
  \newline %
  \Website{open.library.ubc.ca/media/download/pdf/24/1.0421274/4} %
  \\ %
  \Duration{2020}{\Ongoing} & Tannen Acoose, PhD in Mathematics,
  \MYhref{https://www.uregina.ca}{University of Regina}
  (\textbf{Committee Member}) %
  \\ %
  \Year{2018} & Vanessa Braun, MA in Curriculum and Instruction,
  \MYhref{https://www.uregina.ca}{University of Regina}
  (\textbf{External Examiner}) %
  \newline %
  \Website{ourspace.uregina.ca/server/api/core/bitstreams/17ba3eee-6a3b-40aa-add6-4fe95b865431/content} %
  \\ %
  \Duration{2017}{2020} & Shana Graham, PhD in Education,
  \MYhref{https://www.uregina.ca}{University of Regina}
  (\textbf{Committee Member}) %
  \newline %
  \Website{ourspace.uregina.ca/server/api/core/bitstreams/8facba27-f8f1-4d29-b47b-e0d2e7634000/content} %
  \\ %
  \Duration{2012}{2017} & Tannen Acoose, MSc in Mathematics,
  \MYhref{https://www.uregina.ca}{University of Regina}
  (\textbf{Co-supervisor}) %
  \newline %
  \Website{ourspace.uregina.ca/server/api/core/bitstreams/8fe04ddf-4683-467a-837c-220914dcacc0/content} %
  \\ %
  \Duration{2003}{2005} & Meseret Bowden, MSc in Mathematics,
  \MYhref{https://www.uregina.ca}{University of Regina}
  (\textbf{Supervisor}) %
  %\newline
  %\Website{www.uregina.ca/science/mathstat/research/student-theses.html}
\end{EntriesTableDuration}

%%%%%%%%%%%%%%%%%%%%%%%%%%%%%%%%%%%%%%%%%%%%%%%%%%%%%%%%%%%%%%%%%%%%%%%%%%%%%%%
\section{Papers in Refereed Journals}

\begin{EntriesTableYear}
  \Future & Lemon, M., Thom, J.~S., \Me{}., Glanfield, F., \& Nicol,
  C. Time, place, and learning STEM as place.  \textit{Journal of
    Curriculum Studies}.  Revised and submitted. %
  \\ %
  \Future & Dosselmann, R., \Me{}., \& Tayal, V.  Quipu data
  structure.  Submitted to \textit{Springer New Generation
    Computing}. %
  \\ %
  \Future & \Me{}., \& Burns, L. Analysis of a Game Like the Peach
  Stone Bowl Game.  In preparation. %
  \\ %
  \Year{2023} & Nicol, C., Thom, J.~S., \Me{}., Glanfield, F., \&
  Ghostkeeper, E.  (2023) Mathematics education for STEM as place.
  \textit{ZDM --- Mathematics Education}, \textit{55}(7), 1231--1242.
  \DOI{10.1007/s11858-023-01498-z}
  \\
  \Year{2022} & Adusei, K. K., Ng, K. T. W., Karimi, N., Mahmud,
  T. S., \& \Me{}. (2022).  Modeling of municipal waste disposal
  behaviors related to meteorological seasons using recurrent neural
  network LSTM models.  \textit{Ecological Informatics}, \textit{72},
  101925. \DOI{10.1016/j.ecoinf.2022.101925}
  \\
  \Year{2020} & Leung, F.-S., Radzimski, V., \& \Me{}. (2020).
  Reimagining Authentic Mathematical Tasks for Non-STEM Majors.
  \textit{Canadian Journal of Science, Mathematics and Technology
    Education}, \textit{20}(2), 205--217. %
  \newline %
  \DOI{10.1007/s42330-020-00084-9} %
  \\ %
  \Year{2017} & Miller, A. M., \& \Me{}. (2017).  RaráMuri Bird
  Knowledge and Environmental Change in the Sierra Tarahumara,
  Chihuahua, Mexico.  \textit{Journal of Ethnobiology},
  \textit{37}(4), 663--681.
  \DOI{10.2993/0278-0771-37.4.663} %
  \\ %
  \Year{2010} & Kajander, A., Mason, R., Taylor, P., \Me{}., Boland,
  T., Jarvis, D., \& Maciejewski, W.  (2010).  Multiple Visions of
  Teachers’ understandings of Mathematics.  \textit{For the Learning
    of Mathematics}, \textit{30}(3), 50--56. %
  \newline %
  \Website{www.jstor.org/stable/41319540} %
  \\ %
  \Year{2007} & \Me{}., \& Glanfield, F.  (2007).  Balancing equations
  and culture: Indigenous educators reflect on mathematics education.
  \textit{For the Learning of Mathematics}, \textit{27}(3), 27--30.
  \Website{www.jstor.org/stable/40248584} %
  \\ %
  \Year{2006} & Berg, L. C., Longman, S., Hepting, D., \& \Me{}.
  (2006).  Respectful actions in research: Aboriginal adolescents
  speaking their future.  \textit{Delta Kappa Gamma Bulletin},
  \textit{72}(3), 23--29. %
  \\ %
  \Year{2000} & Ferrando, S. E., \Me{}., Bernal, A.~J., \& Bernal,
  L.~J.  (2000).  Probabilistic matching pursuit with Gabor
  dictionaries.  \textit{Signal Processing}, \textit{80}(10),
  2099--2120.  \DOI{10.1016/S0165-1684(00)00071-2}
\end{EntriesTableYear}

%%%%%%%%%%%%%%%%%%%%%%%%%%%%%%%%%%%%%%%%%%%%%%%%%%%%%%%%%%%%%%%%%%%%%%%%%%%%%%%
\section{Book Chapters}

\begin{EntriesTableYear}
  \Future & \Me{}., Graham, S., \& Hughes, A.  Division with
  remainder: Indigenous perspectives.  Under review. %
  \\ %
  \Year{2025} & \Me{}., \& Hughes, A. (2025). Perhaps we didn’t need a
  bridge: In dialogue with Indigenous mathematics.  In K. Kiewitt,
  R. Lutz, G. Cajete, M. do C. dos S. Gonçalves, \& D. K. Johanna
  (Eds.), \textit{Decolonizing Western-Indigenous Dialogues:
    Interwoven Epistemologies for Multiple Modernities}.  Bloomsbury
  Academic. %
  \newline %
  \Website{www.bloomsbury.com/ca/decolonizing-westernindigenous-dialogues-9781350425200} %
  \\ %
  \Year{2018} & \Me{}.  (2018).  Off the Grid.  In S.~Gerofsky (Ed.),
  \textit{Contemporary Environmental and Mathematics Education
    Modelling Using New Geometric Approaches: Geometries of
    Liberation} (pp.~101--121).  Palgrave Pivot.
  \DOI{10.1007/978-3-319-72523-9\_7} %
  \\ %
  \Year{2018} & \Me{}.  (2018).  Foreword.  In A.~Kajander, J.~Holm,
  \& E.~J.~Chernoff (Eds.), \textit{Teaching and Learning Secondary
    School Mathematics Canadian Perspectives in an International
    Context} (1st ed.~2018, pp.~v--xi).  Springer International
  Publishing.  \DOI{10.1007/978-3-319-92390-1} %
\end{EntriesTableYear}

%%%%%%%%%%%%%%%%%%%%%%%%%%%%%%%%%%%%%%%%%%%%%%%%%%%%%%%%%%%%%%%%%%%%%%%%%%%%%%%
\section{Conference Proceedings}

\begin{EntriesTableYear}
  \Future & \Me{} (2024).  ``Mathematics as a Spiritual Being''.
  Invited talk in the \textit{Proceedings of the Fifteenth
    International Conference on Mathematics Education (ICME-15)} %
  \\ %
  \Year{2021} & Staats, S., Ugboajah, I., Chronaki, A., \Me{}, \&
  Sircar, S. (2021). ``There is no America without inequality'':
  Imagining social justice writing in a calculus class.  In
  D.~Kollosche (Ed.), \textit{Exploring new ways to connect:
    Proceedings of the Eleventh International Mathematics Education
    and Society Conference} (Vol. 1, pp. 260--263).  Tredition.
  \DOI{10.5281/zenodo.5393187} %
  \\ %
  \Year{2020} & Gourdeau, F., Deguire, P., LeBlanc, M., \Me{}, Nolan,
  K., Gibara, R., \& Mathieu-Soucy, S.  (2020).  Initiating and
  nurturing collaborations between mathematicians and mathematics
  educators.  In J.~Holm \& S.~Mathieu-Soucy (Eds.),
  \textit{Proceedings of the 2019 Annual Meeting of the Canadian
    Mathematics Education Study Group} (pp. 125–137).  Canadian
  Mathematics Education Study Group. %
  \newline %
  \Website{www.cmesg.org/wp-content/uploads/2021/06/CMESG-2019-website.pdf} %
  \\ %
  \Year{2011} & \Me{}, Lunney Borden, L., \& Wiseman, D.  (2011).  Can
  we be thankful for mathematics?  Mathematical thinking and
  Aboriginal peoples.  In P.~Liljedahl, S.~Oesterle, \& D.~Allan
  (Eds.), \textit{Proceedings of the 2010 Annual Meeting of the
    Canadian Mathematics Education Study Group} (pp. 81--94).
  Canadian Mathematics Education Study Group.  %
  \newline %
  \Website{www.cmesg.org/wp-content/uploads/2015/01/CMESG2010.pdf} %
  \\ %
  \Year{2007} & \Me{}. (2007).  Mathematics as medicine.  In
  P.~Liljedahl (Ed.), \textit{Proceedings of the 2006 Annual Meeting
    of the Canadian Mathematics Education Study Group} (pp. 17--25).
  Canadian Mathematics Education Study Group.  %
  \newline %
  \Website{www.cmesg.org/wp-content/uploads/2015/01/CMESG2006.pdf} %
\end{EntriesTableYear}

%%%%%%%%%%%%%%%%%%%%%%%%%%%%%%%%%%%%%%%%%%%%%%%%%%%%%%%%%%%%%%%%%%%%%%%%%%%%%%%
\section{Invited Talks}

\begin{EntriesTableYear}
  \Future & \Me{}.  \MYhref{https://uwaterloo.ca/hagey-lectures/}{2025
    Hagey Lecture}, \MYhref{https://uwaterloo.ca}{University of
    Waterloo} \Website{uwaterloo.ca/hagey-lectures/} %
  \\ %
  \Year{2025} & Keshen, J, Ottmann, J, Yost, C, \Me{}.
  \MYhref{https://www.uregina.ca/national-building-reconciliation-forum-2025/schedule.html\#fact\_4\_6}{ReconciliAction
    Panel},
  \MYhref{https://www.uregina.ca/national-building-reconciliation-forum-2025/index.html}{National
    Building Reconciliation Forum 2025},
  \MYhref{https://www.fnuniv.ca}{First Nations University of Canada}
  and the \MYhref{https://www.uregina.ca}{University of Regina} %
  \\ %
  \Year{2024} & \Me{}.  Mathematics as a Spiritual Being.
  \MYhref{https://artsandscience.usask.ca/math/index.php}{Department
    of Mathematics and Statistics},
  \MYhref{https://www.usask.ca}{University of Saskatchewan} %
  %\newline %
  \Website{usask.cloud.panopto.eu/Panopto/Pages/Viewer.aspx?id=001c1142-f3a6-476b-bed9-b13c0037c04f} %
  \\ %
  \Year{2023} & \Me{}.
  \MYhref{https://www2.cms.math.ca/Events/winter23/schedule_plenary_prize}{Adrien
    Pouliot Award Prize Lecture}.
  \MYhref{https://cmssmc.wixsite.com/winter23//}{2023 CMS Winter
    Meeting}, \MYhref{https://cms.math.ca/}{Canadian Mathematical
    Society}.  %
  \\ %
  \Year{2023} & \Me{}. Indigenizing University Mathematics. Invited
  plenary speaker at
  \MYhref{https://sites.google.com/mtroyal.ca/amd/plenary-speakers?authuser=0}{Alberta
    Mathematics Dialogue 2023} %
  %\Website{sites.google.com/mtroyal.ca/amd/plenary-speakers?authuser=0}
  %\\
  %\Year{2022} & \Me{}. Indigenous Maths, Global Math, and Indigenizing
  %Mathematics.  Faculty of Mathematics, University of Waterloo.
  %\Website{uwaterloo.ca/math/events/guest-lecture-indigenization-mathematics-dredward-doolittle}
  \\ %
  \Year{2021} & Borwein, N., \& \Me{}.  Indigenous Mathematics.
  Keynote presentation at
  \MYhref{https://carmamaths.org/meetings/ium/}{Indigenising
    University Mathematics 2021},
  \MYhref{https://www.newcastle.edu.au/}{University of Newcastle,
    Australia} %
  %\newline %
  \Website{carmamaths.org/meetings/ium/video/Theme 4 - Edward and
    Naomi.mp4} %
  \\ %
  \Year{2018} & \Me{}.  What is Indigenous Mathematics?
  \MYhref{https://www.uwinnipeg.ca/indigenous/weweni/}{Weweni
    Indigenous Scholars Speaker Series},
  \MYhref{https://www.uwinnipeg.ca/}{University of Winnipeg} %
  %\newline %
  \Website{www.uwinnipeg.ca/indigenous/weweni/weweni-2018/what-is-indigenous-mathematics.html} %
  \\ %
  \Year{2018} & \Me{}.  Geometries of Liberation.  Keynote
  presentation at
  \MYhref{https://indigenous.mathnetwork.educ.ubc.ca/8th-aboriginal-mathematics-symposium/}{Living
    Mathematics in Our Communities: Listening to the Land},
  \MYhref{https://indigenous.mathnetwork.educ.ubc.ca/8th-aboriginal-mathematics-symposium/}{8th
    Aboriginal Mathematics Symposium},
  \MYhref{https://fnhl.ubc.ca/}{First Nations House of Learning},
  \MYhref{https://fnhl.ubc.ca/}{University of British Columbia} %
  \\ %
  \Year{2014} & \Me{}.  Native American Mathematics.
  \MYhref{https://science.ucalgary.ca/mathematics-statistics/about/guy-lecture-series}{Louise
    and Richard~K.~Guy Lecture},
  \MYhref{https://ucalgary.ca}{University of Calgary} %
  \newline %
  \Website{mathtube.org/lecture/video/native-american-mathematics}
  \\ %
  \Year{2014} & \Me{}.  Indigenous Mathematics.  Invited speaker,
  \MYhref{https://www.oise.utoronto.ca/home/sites/default/files/2023-01/connections-2013-14.pdf}{N’gwii
    Kendaasmin (We’ll Learn Together): Drawing on Indigenous
    Knowledges to Transform Teaching and Learning in Mathematics and
    Science}, \MYhref{https://www.oise.utoronto.ca}{Ontario Institute
    for Studies in Education},
  \MYhref{https://www.utoronto.ca}{University of Toronto} %
  \\ %
  \Year{2014} & \Me{}.  Indigenous Mathematics.
  \MYhref{https://news.umanitoba.ca/minding-the-gap-proactive-indigenous-symposium-to-focus-on-maths-and-sciences/}{Indigenous
    Math and Science Symposium},
  \MYhref{https://www.umanitoba.ca}{University of Manitoba} %
  \\ %
  \Year{2006} & \Me{}.  Mathematics as Medicine.  Plenary speaker,
  \MYhref{https://www.cmesg.org/wp-content/uploads/2015/01/CMESG-2006-Program.pdf}{30\textsuperscript{th}
    Annual Meeting of the Canadian Mathematics Education Study Group},
  \MYhref{https://ucalgary.ca}{University of Calgary} %
  \Website{www.cmesg.org/wp-content/uploads/2015/01/CMESG2006.pdf} %
\end{EntriesTableYear}

%%%%%%%%%%%%%%%%%%%%%%%%%%%%%%%%%%%%%%%%%%%%%%%%%%%%%%%%%%%%%%%%%%%%%%%%%%%%%%%
\section{Articles in Periodicals}

\begin{EntriesTableYear}
  \Year{2021} & \Me{} (2021, December).  Explorations in Indigenous
  Mathematics: Drum Lacing.  \textit{Crux Mathematicorum},
  \textit{47}(10), 481--486.
  \Website{cms.math.ca/publications/crux/issue/?volume=47\&issue=10}
  \\
  \Year{2021} & \Me{} (2021, January).  Explorations in Indigenous
  Mathematics: The Starblanket Design.  \textit{Crux Mathematicorum},
  \textit{47}(1), 18--24.
  \Website{cms.math.ca/publications/crux/issue/?volume=47\&issue=1}
  \\
  \Year{2020} & \Me{} (2020, March).  Mathematics and Reconciliation.
  \textit{CMS Notes}, \textit{52}(2), 2--5. %
  \newline %
  \Website{notes.math.ca/en/article/mathematics-and-reconciliation/}
  \\
  \Year{2019} & Barr, D., Desaulniers, S., \Me{}, \& Jungic, V. (2019,
  April).  Indigenization and Reconciliation through University
  Mathematics: Why, When and How?  \textit{CMS Notes}, \textit{51}(2),
  9--11.  \Website{notes.math.ca/archives/Notesv51n2.pdf}
\end{EntriesTableYear}

%%%%%%%%%%%%%%%%%%%%%%%%%%%%%%%%%%%%%%%%%%%%%%%%%%%%%%%%%%%%%%%%%%%%%%%%%%%%%%% 
\section{Peer Review and Editorial Work}

\begin{EntriesTableDuration}
  \Year{2025} & \textbf{Chair},
  \MYhref{https://www.innovation.ca/apply-manage-awards/our-review-process}{Expert
    Committee}, \MYhref{https://www.innovation.ca}{Canada Foundation
    for Innovation (CFI)} \Website{www.innovation.ca} %
  \\ %
  \Duration{2024}{\Ongoing} & \textbf{Guest Editor},
  \MYhref{https://www.mdpi.com/journal/education/about?gad_source=1\&gad_campaignid=22606032410}{\textit{Education
      Sciences}} special issue
  \MYhref{https://www.mdpi.com/journal/education/special_issues/THJA7SLWC2}{Indigenous
    Pedagogies and Perspectives in STEM and Mathematics Education:
    Learning that Supports the Well-Being of Self, Family, Community,
    Land, and Ancestors} %
  \Website{www.mdpi.com/journal/education/special\_issues/THJA7SLWC2} %
  \\ %
  \Year{2024} & \textbf{Referee},
  \MYhref{https://link.springer.com/journal/10763}{\textit{International
      Journal of Science and Mathematics Education}} %
  \Website{springer.com/journal/10763} %
  \\ %
  \Year{2022} & \textbf{Peer Review Member},
  \MYhref{www.researchnet-recherchenet.ca/rnr16/vwOpprtntyDtls.do?prog=3635}{First
    Nations Biobanking and Genomic Research} committee,
  \MYhref{https://cihr-irsc.gc.ca/e/193.html}{Canadian Institutes of
    Health Research (CIHR)}
  \Website{www.researchnet-recherchenet.ca/rnr16/vwOpprtntyDtls.do?prog=3635} %
  \\ %
  \Year{2020} & \textbf{Referee} for the
  \MYhref{https://esj.usask.ca/index.php/esj}{\textit{Engaged Scholar
      Journal: Community-Based Research, Teaching and Learning}}
  special issue on
  \MYhref{https://esj.usask.ca/index.php/esj/issue/view/5160}{Indigenous
    and Trans-Systemic Knowledge Systems} %
  \Website{esj.usask.ca/index.php/esj/issue/view/5160} %
  \\ %
  \Year{2018} & \textbf{Referee} for the
  \MYhref{pubs.lib.umn.edu/index.php/mjum/}{\textit{Minnesota Journal
      of Undergraduate Mathematics}}
  \Website{pubs.lib.umn.edu/index.php/mjum/} %
  \\ %
  \Duration{2016}{2017} & \textbf{Referee},
  \MYhref{https://link.springer.com/journal/42330}{\textit{Canadian
      J.~of Science, Mathematics, and Technology Education}}
  \Website{springer.com/journal/42330} %
  \\ %
  \Year{2014} & \textbf{Referee},
  \MYhref{https://journals.uregina.ca/ineducation}{\textit{in
      education}} journal \Website{journals.uregina.ca/ineducation} %
  \\ %
  \Year{2012} & \textbf{Member},
  \MYhref{https://www.innovation.ca/apply-manage-awards/our-review-process}{Multidisciplinary
    Assessment Committee (MAC)}, \MYhref{https://www.innovation.ca}{Canada
    Foundation for Innovation (CFI)} %
  \newline %
  \Website{www.innovation.ca} %
  \\ %
  \Year{2012} & \textbf{Member}, Insight Grants Selection Committee
  (Aboriginal Research), \MYhref{https://www.sshrc-crsh.gc.ca}{Social
    Sciences and Humanities Research Council (SSHRC)}
  \Website{sshrc-crsh.gc.ca} %
  \\ %
  \Year{2009} & \textbf{Referee},
  \MYhref{https://link.springer.com/journal/42330}{\textit{Canadian
      Journal of Science, Mathematics, and Technology Education}}
  special issue
  \MYhref{https://link.springer.com/journal/42330/volumes-and-issues/9-3}{Indigenous
    Science Education From Place} %: Best Practices on Turtle Island} %
  %\newline %
  \Website{link.springer.com/journal/42330/volumes-and-issues/9-3}
  \\ %
  \Duration{2005}{2009} & \textbf{Reviewer} for
  \MYhref{https://www.pearsoncanadaschool.com/Math/products/mms-wncp-k-9.html}{Math
    Makes Sense --- Pearson WNCP Edition, K--9},
  \MYhref{https://www.pearsoncanadaschool.com/}{Pearson Canada} %
  \newline %
  \Website{www.pearsoncanadaschool.com/Math/products/mms-wncp-k-9.html} %
  \\ %
  \Duration{1996}{1997} & \textbf{Language Editor} for Ivrii,
  V. \textit{Microlocal Analysis and Precise Spectral Asymptotics}.
  Springer. %
\end{EntriesTableDuration}

%%%%%%%%%%%%%%%%%%%%%%%%%%%%%%%%%%%%%%%%%%%%%%%%%%%%%%%%%%%%%%%%%%%%%%%%%%%%%%%
\section{Reports}

\begin{EntriesTableYear}
  \Year{2000} & \Me{}. \textit{Report on the Creation of an Aboriginal
    Studies Course at Queen's University}
\end{EntriesTableYear}

%%%%%%%%%%%%%%%%%%%%%%%%%%%%%%%%%%%%%%%%%%%%%%%%%%%%%%%%%%%%%%%%%%%%%%%%%%%%%%%
\section{Problem Solutions}

\begin{EntriesTableYear}
  \Year{2006} & \Me{} (2006a).  Solution to Problem 3026.  \textit{Crux
  Mathematicorum}, \textit{32}(3), 184--185. %
  \newline %
  \Me{} (2006b).  Solution to Problem 3028.  \textit{Crux Mathematicorum},
  \textit{32}(3), 186--187. %
  \newline %
  \Me{} (2006c).  Solution to Problem 3029.  \textit{Crux Mathematicorum},
  \textit{32}(3), 187--188. %
  \newline %
  \Website{cms.math.ca/publications/crux/issue/?volume=32\&issue=3}
  \\ %
  \Year{2006} & \Me{} (2006d).  Solution to Problem 3080.  \textit{Crux
  Mathematicorum}, \textit{32}(7), 473--475. %
  \newline %
  \Website{cms.math.ca/publications/crux/issue/?volume=32\&issue=7}
  \\ %
  \Year{1988} & \Me{} (1988a). Solution to Problem 1985.305.8.  \textit{Crux
  Mathematicorum}, \textit{14}(2), 42--43. %
  \newline %
  \Me{} (1988b).  Solution to Problem 1985.305.10.  \textit{Crux
  Mathematicorum}, \textit{14}(2), 43--44. %
  \newline %
  \Website{cms.math.ca/publications/crux/issue/?volume=14\&issue=2}
  \\ %
  \Year{1988} & %
  \Me{} (1988c).  Solution to Problem 1985.305.11.  \textit{Crux
  Mathematicorum}, \textit{14}(3), 68--69. %
  \newline %
  \Me{} (1988d).  Solution to Problem 1985.306.19.  \textit{Crux
  Mathematicorum}, \textit{14}(3), 70--71. %
  \newline %
  \Me{} (1988e).  Solution to Problem 1985.306.20.  \textit{Crux
  Mathematicorum}, \textit{14}(3), 71. %
  \newline %
  \Me{} (1988f).  Solution to Problem 1985.307.22.  \textit{Crux
  Mathematicorum}, \textit{14}(3), 71--72. %
  \newline %
  \Me{} (1988g).  Solution to Problem 1985.307.23.  \textit{Crux
  Mathematicorum}, \textit{14}(3), 72. %
  \newline %
  \Me{} (1988h).  Solution to Problem 1985.307.26.  \textit{Crux
  Mathematicorum}, \textit{14}(3), 73--74. %
  \newline %
  \Me{} (1988i).  Solution to Problem 1985.307.27.  \textit{Crux
  Mathematicorum}, \textit{14}(3), 74--75. %
  \newline %
  \Website{cms.math.ca/publications/crux/issue/?volume=14\&issue=3}
  \\
  \Year{1988} & \Me{} (1988j).  Solution to Problem 1986.3.1.  \textit{Crux
  Mathematicorum}, \textit{14}(4), 102--103.  %
  \newline %
  \Me{} (1988k).  Solution to Problem 1986.4.2.  \textit{Crux Mathematicorum},
  \textit{14}(4), 103.  %
  \newline %
  \Me{} (1988l).  Solution to Problem 1986.4.3.  \textit{Crux Mathematicorum},
  \textit{14}(4), 103.  %
  \newline %
  \Me{} (1988m).  Solution to Problem 1986.4.4.  \textit{Crux Mathematicorum},
  \textit{14}(4), 104.  %
  \newline %
  \Website{cms.math.ca/publications/crux/issue/?volume=14\&issue=4}
\end{EntriesTableYear}
      
%%%%%%%%%%%%%%%%%%%%%%%%%%%%%%%%%%%%%%%%%%%%%%%%%%%%%%%%%%%%%%%%%%%%%%%%%%%%%%%
\section{Open Software}

\begin{EntriesTableYear}
  \Year{2025} & \textbf{Curriculum Vitae}, adaptation of a GitHub
  template for developing a readable curriculum vitae in \LaTeX.
  \GitHub{edoolittle/cv}.
\end{EntriesTableYear}

%%%%%%%%%%%%%%%%%%%%%%%%%%%%%%%%%%%%%%%%%%%%%%%%%%%%%%%%%%%%%%%%%%%%%%%%%%%%%%% 
\section{Open Educational Resources}

\begin{EntriesTableYear}
  \Year{2025} & \textbf{Calculus I}, resources including Beamer
  slides, problem sets, and solutions, for use in conjunction with
  Stewart Calculus.  \GitHub{edoolittle/calculus-1}.
\end{EntriesTableYear}

%%%%%%%%%%%%%%%%%%%%%%%%%%%%%%%%%%%%%%%%%%%%%%%%%%%%%%%%%%%%%%%%%%%%%%%%%%%%%%% 
\section{Selected Presentations, Workshops, and Other Media}

\begin{EntriesTableYear}
  \Year{2025} & \Me{}.  Rotations in Indigenous Mathematics and in
  Mathematics Education.  Talk at
  \MYhref{https://www.birs.ca/events/2025/5-day-workshops/25w5472}{The
    Math Bundle}, \MYhref{https://www.birs.ca}{Banff International
    Research Station (BIRS)} workshop
  \MYhref{https://www.birs.ca/events/2025/5-day-workshops/25w5472}{25w5472} %
  %\newline %
  %\Website{www.birs.ca/events/2025/5-day-workshops/25w5472/videos/watch/202507071330-Doolittle.html} %
  \\ %
  \Year{2025} & \Me{}.  Indigenous Students and Mathematics
  Competitions,
  \MYhref{https://www2.cms.math.ca/Events/summer25/sessions\_scientific\#mue}{Math
    Unity: Enhancing Diversity in Mathematics Through Outreach},
  \MYhref{https://summer25.cms.math.ca/}{2025 Canadian Mathematical
    Society Summer Meeting} %
  \\ %
  \Year{2025} & \Me{}.
  \MYhref{https://ssc.ca/en/meeting/annual/presentation/indigenous-games-statistics-education-and-research}{Indigenous
    Games in Statistics Education and Research},
  \MYhref{https://ssc.ca/en/meeting/annual/session/indigenizing-statistics-curriculum}{Indigenizing
    the statistics curriculum},
  \MYhref{https://ssc.ca/en/meetings/annual/2025-ssc-annual-meeting-saskatoon}{2025
    Statistical Society of Canada Annual Meeting}
  \Website{tinyurl.com/indigenous-games-stats-ed-pptx} %
  \\ %
  \Year{2025} & \Me{}.
  \MYhref{https://drive.google.com/file/d/1sM3tuSD0C7mHwZgEpOHXExd5k-6A2t92/view?\#page=32}{Division
    with Remainder: Indigenous Perspectives}.
  \MYhref{https://sites.google.com/view/ucalgary-amd2025/home}{Alberta
    Mathematics Dialogue 2025}, University of Calgary
  \Website{tinyurl.com/division-with-remainder-pptx} %
  \\ %
  \Year{2025} & \Me{}.  Tips and Tricks for UR Courses.  FNUniv
  Instructors Lunch and Learn, \MYhref{https://www.fnuniv.ca/}{First
    Nations University of Canada}
  \Website{urcourses.uregina.ca/mod/page/view.php?id=2932292\&forceview=1}
  % also include one I did on LaTeX, and on Zotero
  \\ %
  \Year{2025} & \Me{}.  Mathematics as Story.
  \MYhref{https://arts-sciences.buffalo.edu/indigenous-studies/news-events/storytellers-conference.html}{Storytelling
    through Art, Language, and Action}.
  \MYhref{www.ub-connect.com/s/1703/alumni/index.aspx?sid=1703\&pgid=5146\&gid=2\&cid=8945\&ecid=8945\&post_id=0}{State
    University of New York at Buffalo}
  % Website is not great ... is there more detail somewhere?
  % https://www.ub-connect.com/s/1703/alumni/index.aspx?sid=1703&pgid=5146&gid=2&cid=8945&ecid=8945&post_id=0
  % Try this
  %\Website{arts-sciences.buffalo.edu/indigenous-studies/news-events/storytellers-conference.html}
  \\
  \Year{2024} & \Me{}.  String Figures and Knots.
  \MYhref{https://atsima.com/}{Aboriginal and Torres Strait Islander
    Mathematics Alliance (ATSIMA)}
  \MYhref{https://atsima.com/stem-steam-camps/}{STEM Camp},
  \MYhref{https://www.birrigai.act.edu.au/}{Birrigai Outdoor School},
  Australian Capital Territory
  %\Website{https://atsima.com/stem-steam-camps/}
  %\\ %
  % TODO: Talk at Newcastle
  % TODO: Talk at University of Queensland
  \\
  \Year{2024} & \Me{}.  Presenter and Organizer for
  \MYhref{https://carmamaths.org/meetings/ium3/}{Indigenising
    University Mathematics 3} international conference held at
  \MYhref{https://www.latrobe.edu.au/}{La Trobe University},
  Melbourne, Australia
  \\
  \Year{2023} & \Me{}.  Better Living Through Combinatorics.  Topic
  session at the 2023 Canadian Math Education Study Group Conference
  \\
  \Year{2023} & \Me{}.  (2023, September 1).  Indigenous Math Podcast
  1 (No. 1) [Broadcast].  \Website{www.cfnuradio.ca/voices/}
  \\
  \Year{2023} & \Me{}, \& Czuy, K. (2023, June 3).  Mathematics is
  Creation, Being, \& Medicine (No. 15) [Broadcast]
  \Website{open.spotify.com/episode/21oDjAQvZIOVWIIRh3CaKS}
  \\
  \Year{2022} & \Me{}. Indigenous Maths, Global Math, and Indigenizing
  Mathematics.  The Centre for Indigenous Knowledges and Languages,
  and the Department of Mathematics and Statistics, York University
  % extremely low quality recording
  \Website{www.youtube.com/watch?v=ptk\_\_Ga43Wg} %
  \\ %
  \Year{2022} & \Me{}. Presenter and Organizer for Indigenizing
  University Mathematics 2 international conference held at the
  University of Newcastle, Australia, and at First Nations University
  of Canada %
  \\ %
  \Year{2022} & \Me{}.  Bridging Indigenous Mathematics and Global
  Mathematics.
  \MYhref{https://event.fourwaves.com/turtleisland2022/}{Turtle Island
    Indigenous Science Conference 2022},
  \MYhref{https://umanitoba.ca}{University of Manitoba} %
  \\ %
  \Year{2022} & \Me{}.  ‘Tent Talk’ on Indigenous Mathematics.
  Teaching and Learning Here and Now conference at the University of
  Regina %
  \\ %
  \Year{2022} & \Me{}, \& Native Stories.  (2022, April 3).  Native
  Stories: Indigenous Mathematicians: Edward Doolittle [Broadcast]
  \Website{nativestories.org/indigenous-mathematicians-edward-doolittle/} %
  \\ %
  \Year{2022} & \Me{}.
  \MYhref{https://researchcentres.wlu.ca/ms2discovery-interdisciplinary-research-institute/news/2022/winter/mathematics-and-reconciliation.html}{Mathematics
    and Reconciliation}.
  \MYhref{https://researchcentres.wlu.ca/ms2discovery-interdisciplinary-research-institute/index.html}{Interdisciplinary
    Research Institute for Mathematical and Statistical Modelling in
    Scientific Discovery, Innovation and Sustainability
    (MS2Discovery)}, \MYhref{https://www.wlu.ca/}{Wilfrid Laurier
    University} %
  \\ %
  \Year{2021} & \Me{}.
  \MYhref{https://drive.google.com/file/d/1xo4\_rcOyYQRDiWs-vaeNS1KRm2TcREFl/view}{Indigenizing
    Mathematics},
  \MYhref{https://www.math.utoronto.ca/~ila/equityforum.html}{University
    of Toronto Mathematics Department Equity Forum} %
  % https://drive.google.com/file/d/1xo4_rcOyYQRDiWs-vaeNS1KRm2TcREFl/view %
  \\ %
  \Year{2021} & Leung, F-S, Radzimski, V, \& \Me{}.  Reimagining
  Authentic Mathematical Tasks Non-STEM Majors,
  \MYhref{http://www.fields.utoronto.ca/activities/21-22/meforum-Nov}{Paper
    Panel Presentation A}, \MYhref{https://fields.utoronto.ca}{Fields}
  \MYhref{http://www.fields.utoronto.ca/activities/workshops/mathed-forum}{MathEd
    Forum}:
  \MYhref{http://www.fields.utoronto.ca/activities/21-22/meforum-Nov}{(Re)imagining
    the M in STEM} %
  \\ %
  \Year{2020} & \Me{}.  Online presentation on Indigenous mathematics
  for Teachers for the Federation of Sovereign Indigenous Nations
  (FSIN), Saskatchewan
  \\
  \Year{2020} & \Me{}. Online presentation for Langara College on
  Indigenizing math education at the post-secondary level
  \\
  \Year{2019} & \Me{}, \& Russell, G.  Mathematics as a Tool for
  Colonization, keynote presentation at Provoking Curriculum, Faculty
  of Education, University of Regina
  %\Website{viewer.joomag.com/education-news-spring-2019/0753087001637684188/p15}
  \\
  \Year{2019} & \Me{}, \& Nolan, K.  Initiating and Nurturing
  Collaborations Between Mathematicians and Mathematics Educators,
  panel presentation at the Canadian Mathematics Education Study Group
  annual meeting
  \\
  \Year{2019} & \Me{}.  Word Puzzles in Indigenous Languages,
  presentation at the Canadian Mathematical Society summer meeting
  \\
  \Year{2019} & \Me{}.  Mathematics of Indigenous Games, a talk for
  high school students at Campbell Collegiate high school in Regina
  \\
  \Year{2019} & \Me{}.  MATH 101 for Indigenous Students, presentation
  at the Canadian Mathematical Society winter meeting %
  \\ %
  \Year{2019} & Leung, F.-S., \Me{}, Zazkis, R., \& Marken, K.
  Looking In, panel presentation at Innovations in New Instructor
  Training, Banff International Research Station workshop 19w2231
  \newline %
  \Website{www.birs.ca/events/2019/2-day-workshops/19w2231/videos/watch/201906220942-Doolittle.html}
  \\ %
  \Year{2018} & Russell, G., Bazzul, J., \Me{}, Donald, D., Higgins,
  M., \& Ji, X.  Exploring Indigenous Spiritualities In-Relation: How
  Might Science and Math Education Become Different?  Panel at the
  \MYhref{https://csse-scee.ca/}{Canadian Society for the Study of
    Education}
  \MYhref{csse-scee.ca/wp-content/uploads/2018/03/2018_CSSE_PC_Prog_CSSE_20180327_Complete.pdf}{XLVI
    Annual Conference}
  \\ %
  \Year{2017} & \Me{}.  Harmonics, Nodal Lines, and Acoustic
  Levitation, keynote presentation at the Treaty 4 Math Fair in Fort
  Qu’appelle, Saskatchewan %
  \\ %
  \Year{2017} & \Me{}.  Transformations, Symmetry, and the
  Starblanket, a talk for Grade 9 students at Campbell Collegiate high
  school in Regina
  \\
  \Year{2016} & \Me{}.  From String Figures to the Fields Medal at the
  6th Aboriginal Students in Math and Science Workshop for Indigenous
  high school students at Simon Fraser University
  \\
  \Year{2016} & \Me{}.  An Exploration of the Fibonacci Sequence at
  Discover Your Direction, an event to encourage Indigenous grade 10
  students to attend university
  \\
  \Year{2016} & \Me{}.  Indigenous String Figures for the Science
  Showcase Series, First Nations University
  \\ %
  \Year{2015} & \Me{}.  Indigenous Students of Mathematics, 17th
  Annual Meeting of Canadian Mathematics Department Chairs %
  \\
  \Year{2015} & CBC \& \Me{}.  “Residential Schools Robbed Edward
  Doolittle of the Mohawk Language.  Then He Reclaimed It.”  CBC
  Radio, June 4, 2015.
  \Website{tinyurl.com/www-cbc-ca-radio-asithappens}
  %\newline
  %\Website{www.cbc.ca/radio/asithappens/as-it-happens-thursday-edition-1.3100783/residential-schools-robbed-edward-doolittle-of-the-mohawk-language-then-he-reclaimed-it-1.3101027}
  \\
  \Year{2015} & \Me{}.  The Development of a Plains Cree Pangrammatic
  Autogram at STEMfest, an international conference, in Saskatoon
  \\
  \Year{2014} & \Me{}.  Cree Syllabic Crosswords, FNUniv Endangered
  Alphabets conference
  \\
  \Year{2014} & \Me{}.  Indigenous Math Education, University of
  Regina/High School Transitions Committee Joint Professional
  Development conference %
  \\
  \Year{2013} & \Me{}.  Mathematics of Planet Earth: Graph Theory of
  the Food Chain for the Science Camp for Aboriginal Youth, First
  Nations University
  \\
  \Year{2013} & \Me{}.  Word Puzzles in Cree at the fourth annual Math
  and Science for Aboriginal Students conference at Simon Fraser
  University
  \\
  \Year{2013} & \Me{}.  Manipulatives in High School Math Education
  series of workshops for the Yorkton Tribal Council %
  \\ %
  \Year{2012} & \Me{}.  Graph Theory in an Indigenous Context, in
  First Nations Math Education, Banff International Research Station
  workshop 12w5076 %
  \newline %
  \Website{www.birs.ca/events/2012/5-day-workshops/12w5076/videos/watch/201211210914-Doolittle.html} %
  \\ %
  \Year{2012} & \Me{}.  Seeing Sounds, Science Camp for Aboriginal
  Youth, First Nations University
  \\
  \Year{2011} & \Me{}.  Mazes and Explorobots for the Science Camp for
  Aboriginal Youth, First Nations University
  \\
  \Year{2010} & \Me{}.  Aboriginal Perspectives in the Saskatchewan
  Mathematics Curriculum, Emerging Professionalism Conference, Faculty
  of Education, University of Regina
  \\
  \Year{2009} & \Me{}.  Music and Signal Processing, presentation to
  Aboriginal K-12 students at the Kehewin Education Institute
  \\
  \Year{2009} & \Me{}.  Teaching Aboriginal Perspectives in High
  School Mathematics at the Sun Country School District Professional
  Development Conference
  %\\ %
  % 2008? Dreamcatching URegina
  \\ %
  \Year{2005} & \Me{}.  Building Community through Science Curriculum
  Actualization, Awasis Conference (organized by the Saskatchewan
  Teachers’ Federation)
  %\\ %
  % 2002/3? CASTS conference Usask
  \\ %
  \Year{2002} & \Me{}.  Knots in Aboriginal Mathematics Education,
  Guest lecture for the Aboriginal Teacher Education Program, Queen’s
  University
  \\
  \Year{1998} & \Me{}.  Issues in the Mathematics Education of
  Aboriginal Students, Engineering Explorations Concordia University
  \\
  \Year{1998} & \Me{}.  Seeing Sounds, Blueprint for the Future,
  National Aboriginal Achievement Foundation
  \\
  \Year{1998} & \Me{}.  Graduation Address, Grand River Post-Secondary
  Education student recognition dinner
  \\
  \Year{1997} & \Me{}.  Experiences of Indigenous Students, Native
  Science Dialogue, University of Toronto
  \\
  \Year{1997} & \Me{}.  Mathematics as a Spiritual Endeavour,
  Aboriginal Youth Career Symposium
  \\
  \Year{1996} & \Me{}.  Introduction of Hopi filmmaker Victor
  Masayesva Jr., Editing Aboriginal Oral Texts: the thirty-second
  annual Conference on Editorial Problems, University of Toronto
  \Website{coilink.org/20.500.12592/xdq1t6}
  \\
  \Year{1995} & \Me{}.  Native Mathematics, Native Issues Seminar,
  First Nations House, University of Toronto
  \\
  \Year{1995} & \Me{}.  Native Mathematics Education, Native Science
  Teachers Camp, University of Toronto
\end{EntriesTableYear}

%%%%%%%%%%%%%%%%%%%%%%%%%%%%%%%%%%%%%%%%%%%%%%%%%%%%%%%%%%%%%%%%%%%%%%%%%%%%%%%
\section{Selected Teaching Experience}

\begin{EntriesTableDuration}
  \Duration{2020}{\Ongoing} & \textbf{Professor}, MATH 110 (Calculus
  I) remote modality (online, synchronous), First Nations University
  of Canada
  \\
  \Duration{2020}{\Ongoing} & \textbf{Mentor}, Putnam Competition
  Training, University of Regina
  % Putnam Training leader for the University of Regina’s Putnam
  % Competition team.  The 2020 competition (written in February 2021,
  % due to COVID) is the first Putnam competition in which the
  % University of Regina participated.  We have since participated in
  % four more Putnam competitions
  %
  % 2001-present A wide cross-section of University of Regina and
  % First Nations University undergraduate mathematics and statistics
  % courses including calculus, linear algebra, differential
  % equations, finite mathematics, adult mathematics, the history of
  % mathematics, and statistics
  \\
  \Year{2023} & \textbf{Sessional Lecturer}, EDPJ/EDJI 1100 (Mathematics
  Education for Primary/Junior/Intermediate) for the Waaban Indigenous
  Education Program at York University
  \\
  \Duration{2013}{2014} & \textbf{Professor}, AMTH 001/091/092 (Adult
  Mathematics) for community-based access programs in Onion Lake and
  Piapot First Nations, First Nations University
  \\
  \Year{2012} & \textbf{Professor}, MATH 101 for the community-based
  Aboriginal Teacher Education Program (Fort Qu’appelle), First
  Nations University
  \\
  % 2009,
  \Year{2018} & \textbf{Professor}, EMTH 215 (Elementary Mathematics
  Education), Department of Indigenous Education, First Nations
  University
  \\
  \Year{2001} & \textbf{Instructor}, Leadership and Management, Grand
  River Polytechnic, Six Nations
  \\
  \Year{2000} & \textbf{Lecturer}, Native Studies, Faculty of
  Environmental Studies, York University
  \\
  \Duration{1999}{2001} & \textbf{Lecturer}, ABS201Y (Aboriginal
  Studies), Faculty of Arts and Science, University of Toronto
  % Responsibilities included course design and delivery for the first
  % and second offerings of ABS201, Introduction to Aboriginal
  % Studies, at the University of Toronto; participation in the
  % selection and supervision of a teaching assistant; and lecturing
  % weekly to a class of 90 students.
  \\
  \Duration{1998}{1999} & \textbf{Instructor}, Elementary Mathematics
  Education, Aboriginal Teacher Education Program, Faculty of
  Education, Queen's University
  % My duties in ATEP included traveling to remote Northern Ontario
  % communities to deliver program material to Ontario Teacher's
  % Certificate candidates, developing curriculum for Aboriginal
  % students and teachers, and grading papers.
  \\
  \Year{1999} & \textbf{Instructor}, Business Math, First Nation
  Management Training and Confederation College, Sioux Lookout
  % In the Confederation College program, I taught a second year
  % college-level course in business mathematics to northern Ontario
  % Native band administrators in the town of Sioux Lookout.  I was
  % responsible for selecting textbooks, developing course outline,
  % producing and marking tests and a final exam, supervising one
  % teaching assistant, and delivering a report on the experience.
  \\
  \Duration{1997}{1998} & \textbf{Lecturer}, MAT 188F/196F (Linear
  Algebra/Calculus), Faculty of Engineering, University of Toronto
  % Duties included lecturing first-year university calculus and
  % linear algebra three times weekly to a class of 80-100 students,
  % testing the students during the term, and supervising two teaching
  % assistants.
  \\
  \Year{1997} & \textbf{Lecturer}, MAT135Y (Calculus I), Faculty of
  Arts and Science, University of Toronto
  % Duties included course coordination, planning lessons, tests and
  % problem sets, scheduling tests, lecturing four times weekly, and
  % supervising three teaching assistants.
  \\
  \Year{1996} & \textbf{Lecturer}, MAT186F (Calculus IB), Faculty of
  Engineering, University of Toronto
  % Duties included lecturing in first-year university calculus three
  % times weekly to a class of 126 students, testing the students during
  % the term, and supervising two teaching assistants.
  % 
  % 1996 Instructor and Mentor, Summer Opportunity in Applied Research,
  % University of Toronto
  % 
  % I taught computing concepts and fractal and dynamical systems
  % programming in C++ and Maple to advanced high school students;
  % devised curriculum; was responsible for computer support on a
  % network of 30 computers; authored for the World Wide Web.
  % 
  % 1996 Instructor, computer course for Aboriginal Legal Services
  % Paralegal Training
  % 
  % I taught elementary computing concepts and basic WordPerfect and
  % Lotus techniques to a group of Native paralegal students; attempted
  % to relate computing concepts to traditional Native learning
  % paradigms.
  % 
  \\
  \Duration{1992}{1994} & \textbf{Academic Counselor}, First Nations
  House, University of Toronto
  \\
  \Duration{1986}{1996} & \textbf{Teaching Assistant}, Department of
  Mathematics, University of Toronto
  % Courses included Complex Analysis, Differential Equations for
  % Engineering Science, Calculus for Life Sciences I and II, Analysis
  % I for Mathematics Specialists, Linear Algebra I for Mathematics
  % Specialists and Calculus for Finance
\end{EntriesTableDuration}


%%%%%%%%%%%%%%%%%%%%%%%%%%%%%%%%%%%%%%%%%%%%%%%%%%%%%%%%%%%%%%%%%%%%%%%%%%%%%%%
\section{Selected Community Service}

\begin{EntriesTableDuration}
  \Duration{2023}{2025} & \textbf{Member},
  \MYhref{https://www.urfa.ca/files/2022-2028-FNUniv-Academic-\%E2\%80\%94-CBA.pdf\#page=156}{Academic
    Performance Review Committee},
  \MYhref{https://www.fnuniv.ca}{First Nations University of Canada}
  \Website{www.fnuniv.ca} %
  \\ %
  \Duration{2023}{2024} & \textbf{Member},
  \MYhref{https://event.fourwaves.com/2024tiisc/pages/8c2de33c-254b-4cc1-a379-dfc8e2621e19}{Local
    Organizing Committee},
  \MYhref{https://event.fourwaves.com/2024tiisc/}{2024 Turtle Island
    Indigenous Science Conference},
  \MYhref{https://www.uregina.ca}{University of Regina} and
  \MYhref{https://www.fnuniv.ca}{First Nations University of Canada} %
  \newline %
  \Website{event.fourwaves.com/2024tiisc/pages/8c2de33c-254b-4cc1-a379-dfc8e2621e19} % 
  \\ %
  \Duration{2022}{2024} & \textbf{Executive Member},
  \MYhref{https://www.cmesg.org}{Canadian Mathematics Education Study
    Group} \Website{www.cmesg.org} %
  \\ %
  \Duration{2021}{\Ongoing} & \textbf{Member}, Canada Jay Mathematical
  Competition committee, Canadian Mathematical Society %
  \newline %
  \Website{cms.math.ca/competitions/cjmc/} %
  \\ %
  \Duration{2020}{2027} & \textbf{Member},
  \MYhref{https://cms.math.ca/about-the-cms/governance/committees/\#rmc}{Mathematics
    and Reconciliation Committee},
  \MYhref{https://cms.math.ca/}{Canadian Mathematical Society} %
  \newline %
  \Website{cms.math.ca/about-the-cms/governance/committees/\#rmc} %
  \\ %
  \Duration{2019}{\Ongoing} & \textbf{Member}, Bargaining Team, First
  Nations University of Canada Academic Bargaining Unit, University of
  Regina Faculty Association \Website{www.urfa.ca} %
  \\ %
  \Duration{2018}{2022} & \textbf{Mentor}, Verna J. Kirkness
  Foundation program \Website{www.vernajkirkness.org} %
  \\ %
  \Year{2018} & \textbf{Judge}, Yakutia International Science Fair in
  Yakutsk, Russia \Website{ysf.lensky-kray.ru/en/} %
  \\ %
  \Duration{2017}{\Ongoing} & \textbf{Judge}, and \textbf{Chair of a
    Judging Group}, Canada Wide Science Fair \Website{cwsf-espc.ca} %
  \\ %
  \Year{2017} & \textbf{Judge}, File Hills Qu'appelle Tribal Council
  Science Fair %
  \\ %
  \Year{2017} & \textbf{Judge}, Treaty 4 Math Fair %
  % \\
  % 2016-2018 Member of Revisioning, Reclaiming, Reconciling School
  % Mathematics, a group of academics, educators, and administrators
  % developing Indigenous math curriculum and policy proposals for the
  % next round of K-12 math curriculum revision in Saskatchewan.
  \\ %
  \Duration{2015}{\Ongoing} & \textbf{Member}, Pension and Benefits
  Committee, First Nations University of Canada
  \Website{www.fnuniv.ca} %
  \\ %
  \Duration{2015}{2017} & \textbf{Secretary/Treasurer}, Native
  Heritage Foundation of Canada %
  \\ %
  \Duration{2013}{2014} & \textbf{Chair}, First Nations Environmental
  Contaminants Program (FNECP) Selection Committee %
  \newline %
  \Website{www.sac-isc.gc.ca/eng/1583779185601/1583779243216}
  \\ %
  \Year{2010} & \textbf{Tax Policy Consultant}, Chiefs of Ontario
  \Website{chiefs-of-ontario.org} %
  \\ %
  \Year{2010} & \textbf{Member}, Academic Reform Task Force, First
  Nations University of Canada \Website{www.fnuniv.ca} %
  \\ %
  \Year{2010} & \textbf{Webmaster}, \textit{Fund First Nations
    University Now!} blog \Website{fnuniv.wordpress.com} %
  \\ %
  \Year{2007} & \textbf{Mentor},
  \MYhref{https://cms.math.ca/competitions/imo/}{Canada’s
    International Math Olympiad} team
  \Website{www.birs.ca/events/2007/summer-schools/07ss005} %
  \\ %
  \Duration{2006}{2008} & \textbf{Member}, Canadian Mathematical
  Olympiad committee \Website{cms.math.ca/competitions/cmo/} %
  \\ %
  \Duration{2003}{2005} & \textbf{Member},
  \MYhref{https://governingcouncil.utoronto.ca/system/files/import-files/a1108-05ii1127.pdf}{President’s
    International Alumni Council},
  \MYhref{https://www.utoronto.ca}{University of Toronto}
  \Website{www.utoronto.ca} %
  \\ %
  \Duration{1999}{2002} & \textbf{Member}, National Aboriginal
  Achievement Foundation Postsecondary Awards Jury %
  %
  % 1997 Named one of the “Great Minds at the University of Toronto”
  % as part of their $300M fundraising campaign
  \\ %
  \Duration{1996}{1997} & \textbf{Graduate Student Representative},
  Aboriginal Advisory Council, University of Toronto
  \Website{www.utoronto.ca} %
  %
  % 1994-1997 Participant in the Native Issues Seminar for Graduate
  % Students and Faculty, First Nations House, University of Toronto
  \\ %
  \Duration{1991}{1992} & \textbf{Member}, Presidential Advisory
  Committee on Race Relations and Anti-Racism Initiatives, University
  of Toronto \Website{www.utoronto.ca} %
  %
  % 1991-1992 Advisory Committee on the Appointment of the Warden of
  % Hart House
  %
  % 1990-1991 Presidential Review of Hart House, University of Toronto
  %
  % 1989-1990 Vice President, Native Students' Association, University
  % of Toronto
  %
  % 1988-1989 Executive Member, Canadian Union of Educational Workers
  % Local 2
  %
  % 1987-1988 President, Mathematics and Statistics Student Union,
  % University of Toronto
\end{EntriesTableDuration}


%%%%%%%%%%%%%%%%%%%%%%%%%%%%%%%%%%%%%%%%%%%%%%%%%%%%%%%%%%%%%%%%%%%%%%%%%%%%%%%
\section{Research Impact: Selected Citations}

\begin{EntriesTableYear}
  \Year{2025} & Lu, J., Si, H., Xu, J., \& Xu, T. (2025).  An overview
  of applications and trends of STEM for learning effectiveness — An
  umbrella review based on 22 meta-analyses.  \textit{Educational
    Research Review}, 100712.  %
  \newline %
  \DOI{10.1016/j.edurev.2025.100712} %
  \\ %
  \Year{2025} & Bakan, G., \& Bircan, M. A. (2025).  Enhancing 21st
  century skills of primary school students in rural areas through
  STEM activities.  \textit{The Journal of Educational Research},
  \textit{0}(0), 1--16. %
  %\newline %
  \DOI{10.1080/00220671.2025.2517265} %
  \\ %
  \Year{2025} & Beumann, S., Weber, D., \& Benölken, R. (2025).
  Identifying and Fostering Giftedness in Students with Disabilities –
  Potential Barriers in Identification and Support from a Mathematics
  Educational Perspective.  \textit{International Journal of Science
    and Mathematics Education}.
  \DOI{10.1007/s10763-025-10587-2} %
  \\ %
  \Year{2025} & Du, W., Cao, Y., Tang, M., Wang, F., \& Wang,
  G. (2025).  Factors influencing AI adoption by Chinese mathematics
  teachers in STEM education.  \textit{Scientific Reports},
  \textit{15}(1), 20429. %
  \DOI{10.1038/s41598-025-06476-x} %
  \\ %
  \Year{2024} & Abtahi, Y., \& Planas, N. (2024).  Mathematics
  teaching and teacher education against marginalisation, or towards
  equity, diversity and inclusion.  \textit{ZDM – Mathematics
    Education}, \textit{56}(3), 307--318. %
  %\newline %
  \DOI{10.1007/s11858-024-01602-x} %
  \\ %
  \Year{2024} & Anderson, J. (2024).  How mathematics in STEM can
  contribute to responsible citizenship education in schools.  In
  J. Anderson \& K. Makar (Eds.), \textit{The Contribution of
    Mathematics to School STEM Education: Current Understandings}
  (pp. 243--256).  Springer Nature.
  \DOI{10.1007/978-981-97-2728-5\_14} %
  \\ %
  \Year{2024} & Loh, K. Q., \& Dasgupta, M. (2024, July 24).  The
  forces of stage design: An interdisciplinary approach to teaching
  normal force, frictional force, and design ethics for non-STEM
  majors.  2023 ASEE Midwest Section Conference.
  \DOI{10.18260/1-2-660.1137-46369} %
  \\ %
  \Year{2024} & Rosa, M., \& Orey, D. C. (2024).  Contributions of the
  pedagogical Action of ethnomodelling to STEM Education.  In
  J. Anderson \& K. Makar (Eds.), \textit{The Contribution of
    Mathematics to School STEM Education: Current Understandings}
  (pp. 277--293).  Springer Nature.
  \DOI{10.1007/978-981-97-2728-5\_16} %
  \\ %
  \Year{2023} & Nordkild, S. I., \& Hætta, O. E. (2023).  Mathematics
  teaching in lávvues from the perspectives of Indigenous education
  and critical peace education.  \textit{Journal of Peace Education},
  \textit{20}(2), 176--195.  %
  \newline %
  \DOI{10.1080/17400201.2023.2206731} %
  \\ %
  \Year{2023} & Ortega-Álvarez, R., \& Casas, A. (2023).  Biocultural
  salient birds: Which biological and cultural factors define them?
  \textit{Frontiers in Conservation Science}, \textit{4}. %
  %\newline %
  \DOI{10.3389/fcosc.2023.1215967} %
  \\ %
  \Year{2023} & Retana-Guiascón, O. G., Santos-Fita, D.,
  Pereyra-Camaal, A., Mejenes-López, S. de M. A., \& Vargas-Soriano,
  J.  (2023).  Conocimiento morfo-anatómico de las aves por mayas
  yucatecos.  \textit{Huitzil}, \textit{24}(1). %
  \newline %
  \DOI{10.28947/hrmo.2023.24.1.720} %
  \\ %
  \Year{2023} & Xenofontos, C., \& Mouroutsou, S. (2023).  Resilience
  in mathematics education research: A systematic review of empirical
  studies.  \textit{Scandinavian Journal of Educational Research},
  \textit{67}(7), 1041--1055. %
  \newline %
  \DOI{10.1080/00313831.2022.2115132} %
  \\ %
  \Year{2022} & Guano, E., \& Moretti, C. (2022).  A tale of two
  ethnographers: Urban anthropologists read Invisible Cities.  In
  B.~Linder (Ed.), \textit{“Invisible Cities” and the Urban
    Imagination} (pp. 117--130).  Springer International Publishing.
  \DOI{10.1007/978-3-031-13048-9\_9} %
  \\ %
  \Year{2022} & Khan, S., \& Bowen, G. M. (2022).  Why Multispecies’
  Flourishing?  \textit{Journal of Research in Science, Mathematics
    and Technology Education}, \textit{5}(1), 1--10.
  \DOI{10.31756/jrsmte.515} %
  \\ %
  \Year{2022} & Rubel, L. H., Herbel-Eisenman, B., Peralta, L. M.,
  Lim, V., Jiang, S., \& Kahn, J. (2022).  Intersectional feminism to
  reenvision mathematical literacies \& precarity.  \textit{Research
    in Mathematics Education}, \textit{24}(2), 224--248.  %
  \newline %
  \DOI{10.1080/14794802.2022.2089908} %
  \\ %
  \Year{2021} & La France, S. (2021).  \textit{Engaging in Speculation
    and Critical Reflection About Future Assessment Practice} [MEd,
  University of Alberta]. %
  %\newline %
  \Website{tinyurl.com/thesis-lafrance-2021-09} %
  %\Website{era.library.ualberta.ca/items/51d1b313-582c-4ea6-9df5-cb929f275c64/view/4c03c021-04b9-435e-be12-b6c94e71eff1/France\_Stephanie\_A\_La\_202109\_MEd.pdf} %
  \\ %
  \Year{2021} & Maciejewski, W. (2021).  Teaching math in real time.
  \textit{Educational Studies in Mathematics}, \textit{108}(1),
  143--159.  \DOI{10.1007/s10649-021-10090-9} %
  \\ %
  \Year{2021} & Meyer, S., \& Aikenhead, G. (2021).  Indigenous
  culture-based school mathematics in action part II: The study’s
  results: what support do teachers need?  \textit{The Mathematics
    Enthusiast}, \textit{18}(1–2), 119--138.  %
  \newline %
  \DOI{10.54870/1551-3440.1517} %
  \\ %
  \Year{2021} & Schiano, B. A. (2021).  \textit{Redesigning
    Developmental Math to Improve Community College Retention Rates
    and Student Success} [EdD, Centenary University]. %
  \newline %
  \Website{www.proquest.com/docview/2524205938/abstract/6D9302DD5E3A4935PQ/1} %
  \\ %
  \Year{2020} & Abtahi, Y. (2020).  The “M” in STEM as a Note of
  Caution: Resilient to What and Responsive to Whose Culture.
  \textit{Canadian Journal of Science, Mathematics, and Technology
    Education}, \textit{20}(2), 281--287. %
  \newline %
  \DOI{10.1007/s42330-020-00093-8} %
  \\ %
  \Year{2020} & Aikenhead, G. (2020).  School Science and Mathematics
  Storylines. \textit{Canadian Journal of Science, Mathematics, and
    Technology Education}, \textit{20}(4), 682--699.
  \DOI{10.1007/s42330-020-00115-5} %
  \\ %
  \Year{2020} & McDougall, D. (2020).  Building Knowledge in a Time of
  Physical Distancing.  \textit{Canadian Journal of Science,
    Mathematics, and Technology Education}, \textit{20}(2), 167--170.
  \DOI{10.1007/s42330-020-00096-5} %
  \\ %
  \Year{2017} & Aikenhead, G. S. (2017).  Enhancing School Mathematics
  Culturally: A Path of Reconciliation.  \textit{Canadian Journal of
    Science, Mathematics, and Technology Education}, \textit{17}(2),
  73--140.  \DOI{10.1080/14926156.2017.1308043} %
  \\ %
  \Year{2017} & Vashchyshyn, I., \& Lunney Borden, L. (2017).
  Spotlight on the profession: In conversation with Dr. Lisa Lunney
  Borden.  \textit{The Variable: An SMTS Periodical}, \textit{2}(5),
  18--23.
  \\ %
  \Year{2016} & Fyhn, A. B., Nutti, Y. J., Nystad, K., Eira, E. J. S.,
  \& Hætta, O. E. (2016).  “We had not dared to do that earlier, but
  now we see that it works”: Creating a culturally responsive
  mathematics exam.  \textit{AlterNative: An International Journal of
    Indigenous Peoples}, \textit{12}(4), 411--424.
  \DOI{10.20507/AlterNative.2016.12.4.6} %
  \\ %
  \Year{2012} & Wagner, D., \& Lunney Borden, L. (2012).  Aiming for
  equity in ethnomathematics research. In B. Herbel-Eisenmann,
  J. Choppin, D. Wagner, \& D. Pimm (Eds.), \textit{Equity in
    Discourse for Mathematics Education: Theories, Practices, and
    Policies} (pp. 69--87).  Springer Netherlands.
  \DOI{10.1007/978-94-007-2813-4\_5} %
  \\ %
  \Year{2011} & Lunney Borden, L. (2011).  The ‘verbification’ of
  mathematics: Using the grammatical structures of Mi’kmaq to support
  student learning.  \textit{For the Learning of Mathematics},
  \textit{31}(3), 8--13.  %
  \newline %
  \Website{flm-journal.org/Articles/2F7403012375137CE62E2DE320F4B.pdf} %
  \\ %
  \Year{2009} & Panina-Beard, N. (2009).  Striving for success:
  Education and career aspiration experiences in the lives of young
  Aboriginal women [MA in Counselling Psychology, Trinity Western
  University].  \Website{tinyurl.com/panina-beard-2009}
  %https://www.researchgate.net/publication/322197256_Striving_for_success_Education_and_career_aspiration_experiences_in_the_lives_of_young_Aboriginal_women
  \\ %
  \Year{1996} & Honsberger, R. (1996).  \textit{From Erdős to Kiev:
    Problems of Olympiad Caliber}.  Mathematical Association of
  America.  \Website{bookstore.ams.org/dol-17} %
\end{EntriesTableYear}


%%%%%%%%%%%%%%%%%%%%%%%%%%%%%%%%%%%%%%%%%%%%%%%%%%%%%%%%%%%%%%%%%%%%%%%%%%%%%%%
\section{Research Impact: Selected Acknowledgments}

\begin{EntriesTableYear}
  \Year{2024} & CEMC. (2024). \textit{Problems with purpose: Volume
    2}.
  \Website{cemc.uwaterloo.ca/resources/problems-with-purpose.php} %
  \\ %
  \Year{2024} & Khan, S., \& Higgins, M. (2024).  In conversation with
  Steven Khan: Sensible and sense-able qualitative literacies for
  multi-species flourishing.  In S. Tolbert, M. F. G. Wallace,
  M. Higgins, \& J. Bazzul (Eds.), \textit{Reimagining Science
    Education in the Anthropocene}, Volume 2 (pp. 389--408).  Springer
  International Publishing.  %
  \newline %
  \DOI{10.1007/978-3-031-35430-4\_21} %
  \\ %
  \Year{2023} & CEMC. (2023).  \textit{Problems with purpose: Volume
    1}.
  \Website{cemc.uwaterloo.ca/resources/problems-with-purpose.php} %
  \\ %
  \Year{2023} & First Nations University of Canada.  (2023, March 24).
  Mohawk Language Program Signing Event [Video recording].
  \Website{www.facebook.com/watch/live/?ref=watch\_permalink\&v=735653604603851}.
  Acknowledgment at 13:30. %
  \\ %
  \Year{2023} & Nolan, K., \& Lunney Borden, L. (2023).  It’s all a
  matter of perspective.  \textit{For the Learning of Mathematics},
  \textit{43}(2), 8--14. %
  %\newline %
  \Website{flm-journal.org/Articles/38B48D80ABF94B2B6AA2C0373439D.pdf}
  \\ %
  \Year{2022} & Medina, M. A. (2022).  \textit{Weaving Indigenous
    mathematics: Ways of sensing, being, and doing} [PhD in Curriculum
  Studies, University of British Columbia].
  \Website{open.library.ubc.ca/media/download/pdf/24/1.0421274/4} \\ %
  \Year{2020} & Graham, S. R. W. (2020).  \textit{Disrupting
    Euro-Western onto-epistemologies (re)imagining possibilities for
    mathematics education through/with Indigenous knowledges and
    complex conversations} [PhD in Education, University of
  Regina].  %
  %\newline %
  \Website{hdl.handle.net/10294/9302} %
  \\ %
  \Year{2018} & Braun, V. K. (2018).  \textit{Beyond the numbers:
    Gaining perspective on the Mathematics Problem towards the
    successful transition of students into university mathematics}
  [MEd in Curriculum and Instruction, Faculty of Graduate Studies and
  Research, University of Regina].
  \Website{hdl.handle.net/10294/8549} %
  \\ %
  \Year{2018} & Hogue, M. (2018).  \textit{Dropping the “T” from
    CAN’T: Enabling Aboriginal post-secondary academic success in
    science and mathematics}.  JCharlton Publishing.  %
  %\newline %
  \Website{tinyurl.com/hogue-dropping-t-from-cant}
  %\Website{www.jcharltonpublishing.com/product/dropping-the-t-from-cant-enabling-aboriginal-post-secondary-academic-success-in-science-and-mathematics/} %
  \\ %
  \Year{2017} & Acoose, T. D. (2017).  \textit{Some probability
    properties of the crack distribution} [MSc in Mathematics, Faculty
  of Graduate Studies and Research, University of Regina].
  \Website{hdl.handle.net/10294/8440} %
  %\\ %
  %\Year{2017} & Wilson, K. (2017).  \textit{Surrender no. 40: Draft
  %  15: A lecture/performance} [Manuscript]. %
  %\newline %
  %\Website{ourspace.uregina.ca/server/api/core/bitstreams/90103324-346a-4b91-aa6a-98db289ba3ec/content} %
  %\\ %
  %\Year{2001} & Freeman, K. (2001).  \textit{Ojibwe women as adult
  %  learners in a teacher education program: Towards an understanding
  %  of Aboriginal women’s experiences of learning and change} [Doctor
  %of Education, University of Toronto].
  %\Website{www.nlc-bnc.ca/obj/s4/f2/dsk3/ftp04/NQ58900.pdf} %
  %\\ %
  %\Year{1999} & Murray, L. J., \& Rice, K. (1999).  \textit{Talking on
  %  the page: Editing Aboriginal oral texts: Papers given at the
  %  thirty-second annual Conference on Editorial Problems}, University
  %of Toronto, 14--16 November 1996 (1st ed.).  University of Toronto
  %Press.  \Website{doi.org/10.3138/9781442680340} %
  \\ %
  \Year{2016} & Russell, G. (2016).  \textit{Valued kinds of knowledge
    and ways of knowing in mathematics and the teaching and learning
    of mathematics: A worldview analysis} [PhD in Curriculum Studies,
  University of Saskatchewan]. %
  \newline %
  \Website{harvest.usask.ca/items/11a8b074-934b-4cb1-ab5a-491a42960462/full} %
  \\ %
  \Year{1998} & Ivrii, V. (1998).  \textit{Microlocal Analysis and
    Precise Spectral Asymptotics}.  Springer Science \& Business
  Media. %
\end{EntriesTableYear}


\end{document}
