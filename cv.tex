%%%%%%%%%%%%%%%%%%%%%%%%%%%%%%%%%%%%%%%%%%%%%%%%%%%%%%%%%%%%%%%%%%%%%%%%%%%%%%%
% A clean template for an academic CV. This is a short summary version.
%
% Uses tabularx to create two column entries (date and job/edu/citation).
% Defines commands to make adding entries simpler.
%
%%%%%%%%%%%%%%%%%%%%%%%%%%%%%%%%%%%%%%%%%%%%%%%%%%%%%%%%%%%%%%%%%%%%%%%%%%%%%%%

\documentclass[9pt,a4paper]{article}

% Useful aliases
\newcommand{\FNUniv}{First Nations University of Canada}
\newcommand{\UofR}{University of Regina}


% Identifying information
\newcommand{\Title}{Curriculum Vit\ae\ Summary}
\newcommand{\FirstName}{Edward}
\newcommand{\MiddleName}{Jon}
\newcommand{\LastName}{Doolittle}
\newcommand{\Initials}{E}
\newcommand{\MyName}{\FirstName\ \MiddleName\ \LastName}
\newcommand{\Me}{\underline{\LastName, \Initials}}  % For citations
\newcommand{\Email}{edoolittle@firstnationsuniversity.ca}
\newcommand{\PersonalWebsite}{www.fnuniv.ca/academic/faculty/dr-edward-doolittle}
\newcommand{\LabWebsite}{www.fnuniv.ca}
\newcommand{\ORCID}{0009-0000-2155-8749}
\newcommand{\GitHubProfile}{edoolittle}

% Names for citing coauthors
\newcommand{\Shalene}{Jobin, S}
\newcommand{\Shaun}{Fallat, S}
\newcommand{\Layne}{Burns, L}
\newcommand{\Cynthia}{Nicol, C}
\newcommand{\Florence}{Glanfield, F}
\newcommand{\Jennifer}{Thom, J}
\newcommand{\Malabika}{Pramanik, M}
\newcommand{\Laleh}{Behjat, L}
\newcommand{\Marc}{Frappier, M}
\newcommand{\Viqar}{Husain, V}
\newcommand{\Mark}{Lewis, M}
\newcommand{\Lei}{Sun, L}
\newcommand{\Betty}{McKenna, B}
\newcommand{\Dwayne}{Donald, D}
\newcommand{\Gladys}{Sterenberg G}


% Load packages
%%%%%%%%%%%%%%%%%%%%%%%%%%%%%%%%%%%%%%%%%%%%%%%%%%%%%%%%%%%%%%%%%%%%%%%%%%%%%%%

% Full Unicode support for non-ASCII characters
\usepackage[utf8]{inputenc}
\usepackage[english]{babel}
\usepackage[TU]{fontenc}

% Set main fonts
\usepackage[sfdefault]{atkinson}
\usepackage[ttdefault]{sourcecodepro}

% Icon fonts
\usepackage{fontawesome5}
\usepackage{academicons}

% Disable hyphenation
\usepackage[none]{hyphenat}

% Control the font size
\usepackage{anyfontsize}

% For fancy and multipage tables
\usepackage{tabularx}
\usepackage{ltablex}

% For new environments
\usepackage{environ}

% Manage dates and times
\usepackage{datetime}

% Set the page margins
\usepackage{geometry}

% To get the total page numbers (\pageref{LastPage})
\usepackage{lastpage}

% Control spacing in enumerates
\usepackage{enumitem}

% Use custom colors
\usepackage[usenames,dvipsnames]{xcolor}

% Configure section titles
\usepackage[nobottomtitles*]{titlesec}
\renewcommand{\bottomtitlespace}{.1\textheight}

% Fancy header configuration
\usepackage{fancyhdr}

% Control PDF metadata and links
\usepackage[colorlinks=true]{hyperref}

% This should break long URLs per
% https://tex.stackexchange.com/questions/386495/how-to-wrap-a-url-or-reference
% but does not seem to help
%% Wrap long links
%\usepackage[ocgcolorlinks]{ocgx2}

% Template configuration
%%%%%%%%%%%%%%%%%%%%%%%%%%%%%%%%%%%%%%%%%%%%%%%%%%%%%%%%%%%%%%%%%%%%%%%%%%%%%%%

\geometry{%
  margin=12.5mm,
  headsep=1mm,
  headheight=0mm,
  footskip=5mm,
  includehead=true,
  includefoot=true
}

% Custom colors
\definecolor{mediumgray}{gray}{0.5}
\definecolor{lightgray}{gray}{0.9}
\definecolor{mediumblue}{HTML}{2060c2}
\definecolor{lightblue}{HTML}{a0c3ff}
\definecolor{darkblue}{HTML}{000080}

% No indentation
\setlength\parindent{0cm}

% Increase the line spacing
\renewcommand{\baselinestretch}{1.1}
% and the spacing between rows in tables
\renewcommand{\arraystretch}{1.25}

% Remove space between items in itemize and enumerate
\setlist{nosep}

% Set the spacing and format of sections
\titleformat{\section}
  {\normalfont\Large\mdseries} % format
  {} % label
  {0pt} % separation (left separation for hang)
  {} % text before title
  [\titlerule] % text after title
\titlespacing*{\section}
  {0pt} % left pad
  {0.1cm} % before
  {0cm} % after

% Disable number of sections. Use this instead of "section*" so that
% the sections still appear as PDF bookmarks. Otherwise, would have
% to add the table of contents entries manually.
\makeatletter
\renewcommand{\@seccntformat}[1]{}
\makeatother

% Define a new environment to place all CV entries in a 2-column table.
% Left column are the dates, right column the entries.
\newcommand{\TablePad}{\vspace{-0.2cm}}
\NewEnviron{EntriesTableDuration}{
\TablePad
\begin{tabularx}{\textwidth}{@{}p{0.10\textwidth}@{\hspace{0.02\textwidth}}p{0.88\textwidth}@{}}
  \BODY
\end{tabularx}
\TablePad
}
\NewEnviron{EntriesTableYear}{
\TablePad
\begin{tabularx}{\textwidth}{@{}p{0.05\textwidth}@{\hspace{0.01\textwidth}}p{0.94\textwidth}@{}}
  \BODY
\end{tabularx}
\TablePad
}

% Macros to set the year and duration on the left column
\newcommand{\Duration}[2]{\fontsize{10pt}{0}\selectfont \texttt{#1-#2}}
\newcommand{\Year}[1]{\fontsize{10pt}{0}\selectfont \texttt{#1}}
\newcommand{\Ongoing}{on}
\newcommand{\Future}{future}

% Macros to add links and mark publications
\newcommand{\DOI}[1]{doi:\href{https://doi.org/#1}{#1}}
\newcommand{\Website}[1]{\href{https://#1}{#1}}
\newcommand{\Preprint}[1]{Preprint: \href{https://doi.org/#1}{#1}}
\newcommand{\GitHub}[1]{\faGithub{} \href{https://github.com/#1}{#1}}
\newcommand{\Data}[1]{\faChartBar{} doi:\href{https://doi.org/#1}{#1}}
\newcommand{\MYhref}[3][darkblue]{\href{#2}{\color{#1}{#3}}}

% Define command to insert month name and year as date
\newdateformat{monthyear}{\monthname[\THEMONTH], \THEYEAR}

% Configure a fancy footer
\newcommand{\Separator}{\hspace{3pt}|\hspace{3pt}}
\newcommand{\FooterFont}{\footnotesize\color{mediumgray}}
\pagestyle{fancy}
\fancyhf{}
\lfoot{%
  \FooterFont{}
  \MyName{}
  \Separator{}
  \Title{}
}
\rfoot{%
  \FooterFont{}
  Last updated: \monthyear\today{}
  \Separator{}
  \thepage\space of\space \pageref*{LastPage}
}
\renewcommand{\headrulewidth}{0pt}
\renewcommand{\footrulewidth}{1pt}
\preto{\footrule}{\color{lightgray}}

% Metadata for the PDF output and control of hyperlinks
\hypersetup{
  colorlinks,
  allcolors=mediumblue,
  breaklinks=true,
  pdftitle={\Title{} - \MyName},
  pdfauthor={\MyName},
}
% This should break long URLs per
% https://tex.stackexchange.com/questions/386495/how-to-wrap-a-url-or-reference
% but does not seem to help
%\makeatletter
%    \g@addto@macro{\UrlBreaks}{\do\/\do\-\do\_}
%\makeatother

%%%%%%%%%%%%%%%%%%%%%%%%%%%%%%%%%%%%%%%%%%%%%%%%%%%%%%%%%%%%%%%%%%%%%%%%%%%%%%%
\begin{document}

\begin{minipage}[t]{0.5\textwidth}
  {\fontsize{20pt}{0}\selectfont\MyName}
\end{minipage}
\begin{minipage}[t]{0.5\textwidth}
  \begin{flushright}
    \Title{}
  \end{flushright}
\end{minipage}
\\[-0.1cm]
\textcolor{lightgray}{\rule{\textwidth}{3pt}}
\begin{minipage}[t]{0.5\textwidth}
  ORCID: \href{https://orcid.org/\ORCID}{\ORCID}
  \\
  Email: \href{mailto:\Email}{\Email}
  \\
  Website: \Website{\PersonalWebsite}
  %\\
  %Research Group: \Website{\LabWebsite}
\end{minipage}
\begin{minipage}[t]{0.5\textwidth}
  \begin{flushright}
  Associate Professor of Mathematics
  \\
  Departament of Indigenous Knowledges and Science
  \\
  First Nations University of Canada
  \\
  1 First Nations Way, Regina, Saskatchewan, S4S7K2, Canada
  \end{flushright}
\end{minipage}
\vspace{0.3cm}

%%%%%%%%%%%%%%%%%%%%%%%%%%%%%%%%%%%%%%%%%%%%%%%%%%%%%%%%%%%%%%%%%%%%%%%%%%%%%%%
\section{Current Professional Appointments}

\begin{EntriesTableDuration}
  \Duration{2024}{\Ongoing} & \textbf{Associate Dean, Research and
    Graduate Programs}, \FNUniv, \Website{www.fnuniv.ca}
  \\
  \Duration{2024}{\Ongoing} & \textbf{Co-chair (Indigenous Research)},
  Research Ethics Board, \UofR, \Website{www.uregina.ca}
  \\
  \Duration{2014}{\Ongoing} & \textbf{Associate Professor of
    Mathematics}, \FNUniv, \Website{www.fnuniv.ca}
\end{EntriesTableDuration}

%%%%%%%%%%%%%%%%%%%%%%%%%%%%%%%%%%%%%%%%%%%%%%%%%%%%%%%%%%%%%%%%%%%%%%%%%%%%%%%
\section{Previous Professional Appointments}

\begin{EntriesTableDuration}
  \Duration{2011}{2014} & \textbf{Associate Professor of Mathematics
    and Department Head}, \FNUniv %\Website{www.fnuniv.ca}
  \\
  \Duration{2009}{2011} & \textbf{Associate Professor of Mathematics},
  \FNUniv\ \Website{www.fnuniv.ca}
  \\
  \Duration{2008}{2009} & \textbf{Assistant Professor of Mathematics},
  \FNUniv\ \Website{www.fnuniv.ca}
  \\
  \Duration{2005}{2008} & \textbf{Assistant Professor of Mathematics},
  \UofR\ \Website{www.uregina.ca}
  \\
  \Duration{2002}{2005} & \textbf{Assistant Professor of Mathematics
    and Department Head}, \FNUniv %\Website{www.fnuniv.ca}
  \\
  \Duration{2001}{2002} & \textbf{Assistant Professor of Mathematics},
  Saskatchewan Indian Federated College
\end{EntriesTableDuration}

%%%%%%%%%%%%%%%%%%%%%%%%%%%%%%%%%%%%%%%%%%%%%%%%%%%%%%%%%%%%%%%%%%%%%%%%%%%%%%%
\section{Leadership and Involvement}

\begin{EntriesTableDuration}
  \Duration{2023}{\Ongoing} & \textbf{Chair}, Mathematics and
  Reconciliation Committee, Canadian Mathematical Society \Website{cms.math.ca}
  \\
  \Duration{2022}{\Ongoing} & \textbf{Board Member}, Board of Equity,
  Diversity and Inclusion, Fields Institute for Research in
  Mathematical Sciences \Website{www.fields.utoronto.ca}
  \\
  \Duration{2021}{\Ongoing} & \textbf{Committee Member}, Indigenous
  Engagement Committee, Pacific Institute for the Mathematical
  Sciences \Website{www.pims.math.ca/people/committees/indigenous-engagement-committee}
  \\
  \Duration{2020}{\Ongoing} & \textbf{Board Member}, Board of Equity,
  Diversity, and Inclusion, Banff International Research Station
  \Website{www.birs.ca/about/governance/scientific-management/Equity-Diversity-Inclusion-Board/current-members-EDIB}
  \\
  \Duration{2020}{2023} & \textbf{Committee Member}, Mathematics and
  Reconciliation Committee, Canadian Mathematical Society
  %\Website{cms.math.ca}
  \\
  \Duration{2019}{\Ongoing} & \textbf{Judge}, National Judging Team,
  Youth Science Canada \Website{youthscience.ca}
  \\
  \Duration{2018}{\Ongoing} & \textbf{Trustee}, Trust Fund Committee,
  University of Regina Faculty Association \Website{www.urfa.ca}
\end{EntriesTableDuration}

%%%%%%%%%%%%%%%%%%%%%%%%%%%%%%%%%%%%%%%%%%%%%%%%%%%%%%%%%%%%%%%%%%%%%%%%%%%%%%% 
\section{Education}

\begin{EntriesTableDuration}
  \Duration{1992}{1997} & \textbf{PhD in Mathematics}, University of
  Toronto \Website{www.utoronto.ca}
  \newline
  \textbf{Thesis:} A Parametrix for Stable Step Two Hypoelliptic
  Partial Differential Operators
  \textbf{Advisor:} Peter Greiner
  \newline
  \Website{utoronto.scholaris.ca/server/api/core/bitstreams/a6c7bee0-0bce-49dd-8044-e0695623a0dc/content}
  \\
  \Duration{1990}{1992} & \textbf{MSc in Mathematics}, University of
  Toronto, \Website{www.utoronto.ca}
  \\
  \Duration{1985}{1990} & \textbf{BSc in Mathematics}, University of
  Toronto, \Website{www.utoronto.ca}
\end{EntriesTableDuration}

%%%%%%%%%%%%%%%%%%%%%%%%%%%%%%%%%%%%%%%%%%%%%%%%%%%%%%%%%%%%%%%%%%%%%%%%%%%%%%%
\section{Fellowships and Awards}

\begin{EntriesTableYear}
  \Year{2024} & \textbf{Fellow} of the Canadian Mathematical Society
  \Website{cms.math.ca/awards/fellows-of-the-cms/}
  \\
  \Year{2023} & \textbf{Adrien Pouliot Award}, Canadian Mathematical
  Society \Website{cms.math.ca/awards/adrien-pouliot-award/}
  \\
  \Year{1992} & \textbf{Governor General's Gold Medal}
  \Website{www.gg.ca/en/honours/recipients/116-5733}
  %\\
  %\Year{1987} & \textbf{Honorable Mention}, William Lowell Putnam
  %Mathematical Competition \newline
  %\Website{kskedlaya.org/putnam-archive/putnam1987results.html}
\end{EntriesTableYear}

%%%%%%%%%%%%%%%%%%%%%%%%%%%%%%%%%%%%%%%%%%%%%%%%%%%%%%%%%%%%%%%%%%%%%%%%%%%%%%%
\section{Indigenous Identity}

\begin{EntriesTableDuration}
  & I am a \textbf{Status Indian}, member of the \textbf{Lower Mohawk}
  band of \textbf{Six Nations}.
\end{EntriesTableDuration}

%%%%%%%%%%%%%%%%%%%%%%%%%%%%%%%%%%%%%%%%%%%%%%%%%%%%%%%%%%%%%%%%%%%%%%%%%%%%%%% 
\section{Grants}

\begin{EntriesTableDuration}
  \Year{2025} & \textbf{The Math Bundle}.  Banff International
  Research Station workshop 25w5472.  \newline \Me{} (PI),
  \Florence{}, \Betty{}
  \Website{www.birs.ca/events/2025/5-day-workshops/25w5472}
  \\
  \Duration{2024}{2030} & \textbf{Critical Approaches to Indigenous
    Relationality}.  SSHRC Partnership Grant.  \newline \Shalene{}
  (PI), \Me{}, et.~al (\$2.5 million over 6 years) \newline
  \Website{www.sshrc-crsh.gc.ca/results-resultats/recipients-recipiendaires/2023/pg-sp-eng.aspx}
  \\
  \Duration{2024}{2026} & \textbf{The Mathematics of Indigenous
    Games}.  First Nations University Board of Governors Grant.
  \newline \Me{} (PI), \Shaun{}, \Layne{} (\$5000 over 2 years)
  \\
  \Duration{2023}{2028} & \textbf{Mathematics Education for STEM as
    Place}.  SSHRC Insight Grant.  \newline \Cynthia{} (PI), \Me{},
  \Florence{}, \Jennifer{} (\$360,000 over 5 years) \newline
  \Website{www.sshrc-crsh.gc.ca/results-resultats/recipients-recipiendaires/2021/ig-ss-eng.aspx}
  \\
  \Duration{2022}{2027} & \textbf{Banff International Research
    Station}.
  \MYhref{https://www.nserc-crsng.gc.ca/professors-professeurs/Grants-Subs/DIS-ADIR_eng.asp}{NSERC Discovery Institutes Support Grants}.
  %\newline
  \newline
  \Malabika{} (PI), \Laleh{}, \Me{}, \Marc{}, \Viqar{}, \Mark{}, \Lei{}.
  (\$5 million over 5 years)
  \newline
  \Website{www.nserc-crsng.gc.ca/ase-oro/Details-Detailles\_eng.asp?id=767960}
  \newline 
  \Website{www.nserc-crsng.gc.ca/ase-oro/Details-Detailles\_eng.asp?id=758456}
  \\
  \Year{2013} & \textbf{Understanding Relationships between Aboriginal
    Knowledge Systems, Wisdom Traditions, and Mathematics: Research
    Possibilities}.  BIRS Workshop 13w5120.
  \newline
  \Me{}; \Florence{}
  \Website{www.birs.ca/events/2013/5-day-workshops/13w5120}
  \\
  \Duration{2009}{2011} & \textbf{Creating a Research Network to
    Develop an Understanding of Relationships between Aboriginal
    Knowledge Systems, Wisdom Traditions, and Mathematics Education}
  \newline
  SSHRC Aboriginal Research Development Grant.
  \newline
  \Florence{} (PI), \Dwayne{}, \Me{}, \Gladys{} (\$25,000 over 2 years)
  \newline
  \Website{www.outil.ost.uqam.ca/crsh/Detail.aspx?Cle=78132\&Langue=2}
  \\
  \Year{2008} & \textbf{MATH 104/105 Calculus IA/Calculus IB}
  \newline
  Technology Enhanced Learning grant, Ministry of Advanced Education,
  Saskatchewan.
  \newline
  Herman, Allen (PI) \& \Me{}. (\$40,000)
  % \\
  % \Duration{2005}{2010} & \textbf{Understanding the Dynamics of Risk and
  %   Protective Factors in Promoting Success in Science and Mathematics}
  % \newline
  % NSERC CRYSTAL Grant.
  % \newline
  % Robinson, G. (PI), et. al. \& \Me{} (Collaborator)
  % (\$1,000,000 over 5 years)
  % \Website{www.nserc-crsng.gc.ca/ase-oro/index\_eng.asp?new}
  \\
  \Duration{1991}{1995} & \textbf{NSERC Postgraduate Scholarship}
  (\$56,000 over 4 years)
  \newline
  \Website{www.nserc-crsng.gc.ca/ase-oro/index\_eng.asp?new}
\end{EntriesTableDuration}

%%%%%%%%%%%%%%%%%%%%%%%%%%%%%%%%%%%%%%%%%%%%%%%%%%%%%%%%%%%%%%%%%%%%%%%%%%%%%%% 
\section{Supervision}

\begin{EntriesTableDuration}
  \Duration{2024}{\Ongoing} & Layne Burns, MSc in Mathematics,
  University of Regina (\textbf{Co-supervisor})
  \\
  \Duration{2024}{\Ongoing} & Whitney Ogle, MA in
  Indigenous Education,
  University of Regina (\textbf{Committee Member})
  \\
  \Duration{2023}{2024} & Layne Burns, NSERC
  \MYhref{https://www.nserc-crsng.gc.ca/students-etudiants/ug-pc/usra-brpc_eng.asp}{USRA},
  First Nations University of Canada (\textbf{Supervisor})
  \\
  \Duration{2023}{\Ongoing} & John Porrit, MA in Mathematics
  Education, University of Regina (\textbf{Co-supervisor})
  \\
  \Duration{2023}{\Ongoing} & Ehdaa Matia, PhD in Mathematics
  Education, University of Regina (\textbf{Committee Member})
  \\
  \Duration{2022}{2023} & Shana Graham, Postdoctoral Fellowship,
  University of Regina (\textbf{Co-supervisor})
  \\
  \Year{2022} & Myron Medina, PhD in Mathematics Education, University
  of British Columbia (\textbf{External Examiner})
  \newline
  \Website{open.library.ubc.ca/media/download/pdf/24/1.0421274/4}
  \\
  \Duration{2020}{\Ongoing} & Tannen Acoose, PhD in Mathematics,
  University of Regina (\textbf{Committee Member})
  \\
  \Year{2018} & Vanessa Braun, MA in Mathematics Education, University
  of Regina (\textbf{External Examiner})
  \newline
  \Website{ourspace.uregina.ca/server/api/core/bitstreams/17ba3eee-6a3b-40aa-add6-4fe95b865431/content}
  \\
  \Duration{2017}{2020} & Shana Graham, PhD in Mathematics Education,
  University of Regina (\textbf{Committee Member})
  \newline
  \Website{ourspace.uregina.ca/server/api/core/bitstreams/8facba27-f8f1-4d29-b47b-e0d2e7634000/content}
  \\
  \Duration{2012}{2017} & Tannen Acoose, MSc in Mathematics,
  University of Regina (\textbf{Co-supervisor})
  \newline
  \Website{ourspace.uregina.ca/server/api/core/bitstreams/8fe04ddf-4683-467a-837c-220914dcacc0/content}
  \\
  \Duration{2003}{2005} & Meseret Bowden, MSc in Mathematics,
  University of Regina (\textbf{Supervisor})
  %\newline
  %\Website{www.uregina.ca/science/mathstat/research/student-theses.html}
\end{EntriesTableDuration}

%%%%%%%%%%%%%%%%%%%%%%%%%%%%%%%%%%%%%%%%%%%%%%%%%%%%%%%%%%%%%%%%%%%%%%%%%%%%%%%
\section{Papers in Refereed Journals}

\begin{EntriesTableYear}
  \Future & Lemon, M, \Me{}, Glanfield, F, Nicol, C, Thom, J.~S. (2025).
  Time, place, and learning STEM as place.  Journal of Curriculum Studies.
  Under review.
  \\
  \Year{2023} & Nicol, C., Thom, J.~S., \Me{}, Glanfield, F., \&
  Ghostkeeper, E.  (2023) Mathematics education for STEM as place.
  ZDM --- Mathematics Education, 55(7), 1231--1242.
  \Website{doi.org/10.1007/s11858-023-01498-z}
  \\
  \Year{2022} & Adusei, K. K., Ng, K. T. W., Karimi, N., Mahmud,
  T. S., \& \Me{} (2022).  Modeling of municipal waste disposal
  behaviors related to meteorological seasons using recurrent neural
  network LSTM models.  Ecological Informatics, 72,
  101925. \Website{doi.org/10.1016/j.ecoinf.2022.101925}
  \\
  \Year{2020} & Leung, F.-S., Radzimski, V., \& \Me{} (2020).
  Reimagining Authentic Mathematical Tasks for Non-STEM Majors.
  Canadian Journal of Science, Mathematics and Technology Education,
  20(2), 205--217.  \Website{doi.org/10.1007/s42330-020-00084-9}
  \\
  \Year{2017} & Miller, A. M., \& \Me{} (2017).  RaráMuri Bird
  Knowledge and Environmental Change in the Sierra Tarahumara,
  Chihuahua, Mexico.  Journal of Ethnobiology, 37(4), 663–-681.
  \Website{doi.org/10.2993/0278-0771-37.4.663}
  \\
  \Year{2010} & Kajander, A., Mason, R., Taylor, P., \Me{}, Boland,
  T., Jarvis, D., \& Maciejewski, W.  (2010).  Multiple Visions of
  Teachers’ understandings of Mathematics.  For the Learning of
  Mathematics, 30(3), 50--56.  \Website{www.jstor.org/stable/41319540}
  \\
  \Year{2007} & \Me{}, \& Glanfield, F.  (2007).  Balancing equations
  and culture: Indigenous educators reflect on mathematics education.
  For the Learning of Mathematics, 27(3), 27--30.
  \Website{www.jstor.org/stable/40248584}
  \\
  \Year{2006} & Berg, L. C., Longman, S., Hepting, D., \& \Me{}
  (2006).  Respectful actions in research: Aboriginal adolescents
  speaking their future.  DELTA KAPPA GAMMA BULLETIN, 72(3), 23--29.
  \\
  \Year{2000} & Ferrando, S. E., \Me{}, Bernal, A. J., \&
  Bernal, L. J.  (2000).  Probabilistic matching pursuit with Gabor
  dictionaries.  Signal Processing, 80(10), 2099--2120.
  \Website{doi.org/10.1016/S0165-1684(00)00071-2}
\end{EntriesTableYear}

%%%%%%%%%%%%%%%%%%%%%%%%%%%%%%%%%%%%%%%%%%%%%%%%%%%%%%%%%%%%%%%%%%%%%%%%%%%%%%%
\section{Book Chapters}

\begin{EntriesTableYear}
  \Future & \Me{}, Graham, S., \& Hughes, A. (2025).  Division with
  remainder: Indigenous perspectives.
  \\
  \Future & \Me{}, \& Hughes, A. (2025).  Perhaps we didn’t need a
  bridge: In dialogue with Indigenous mathematics.  In Decolonizing
  Western-Indigenous Dialogues: Interwoven Epistemologies for Multiple
  Modernities. Bloomsbury Academic.
  \\
  \Year{2018} & \Me{} (2018).  Off the Grid.  In S.~Gerofsky (Ed.),
  Contemporary Environmental and Mathematics Education Modelling Using
  New Geometric Approaches (pp.~101–-121).  Springer International
  Publishing AG.  \Website{doi.org/10.1007/978-3-319-72523-9\_7}
  \\
  \Year{2018} & \Me{} (2018).  Foreword.  In A.~Kajander, J.~Holm, \&
  E.~J.~Chernoff (Eds.), Teaching and Learning Secondary School
  Mathematics Canadian Perspectives in an International Context (1st
  ed. 2018, pp. v–-xi).  Springer International Publishing.
  \Website{doi.org/10.1007/978-3-319-92390-1}
\end{EntriesTableYear}

%%%%%%%%%%%%%%%%%%%%%%%%%%%%%%%%%%%%%%%%%%%%%%%%%%%%%%%%%%%%%%%%%%%%%%%%%%%%%%%
\section{Articles in Periodicals}

\begin{EntriesTableYear}
  \Year{2021} & \Me{} (2021, December).  Explorations in Indigenous
  Mathematics: Drum Lacing.  Crux Mathematicorum, 47(10), 481-–486.
  \Website{cms.math.ca/publications/crux/issue/?volume=47\&issue=10}
  \\
  \Year{2021} & \Me{} (2021, January).  Explorations in Indigenous
  Mathematics: The Starblanket Design.  Crux Mathematicorum, 47(1),
  18-–24.
  \Website{cms.math.ca/publications/crux/issue/?volume=47\&issue=1}
  \\
  \Year{2020} & \Me{} (2020, March).  Mathematics and Reconciliation.
  CMS Notes, 52(2), 2–-5.
  \newline
  \Website{notes.math.ca/en/article/mathematics-and-reconciliation/}
  \\
  \Year{2019} & Barr, D., Desaulniers, S., \Me{}, \& Jungic, V. (2019,
  April).  Indigenization and Reconciliation through University
  Mathematics: Why, When and How?  CMS Notes, 51(2), 9--11.
  \Website{notes.math.ca/archives/Notesv51n2.pdf}
\end{EntriesTableYear}

%%%%%%%%%%%%%%%%%%%%%%%%%%%%%%%%%%%%%%%%%%%%%%%%%%%%%%%%%%%%%%%%%%%%%%%%%%%%%%%
\section{Conference Proceedings}

\begin{EntriesTableYear}
  \Future & \Me{} (2024).  ``Mathematics as a Spiritual Being''.
  Invited talk in the Proceedings of the Fifteenth International
  Conference on Mathematics Education (ICME-15)
  \\
  \Year{2021} & Staats, S., Ugboajah, I., Chronaki, A., \Me{}, \&
  Sircar, S. (2021). ``There is no America without inequality'':
  Imagining social justice writing in a calculus class.  In
  D.~Kollosche (Ed.), Exploring new ways to connect: Proceedings of
  the Eleventh International Mathematics Education and Society
  Conference (Vol. 1, pp. 260-–263).  Tredition.
  \Website{doi.org/10.5281/zenodo.5393187}
  \\
  \Year{2010} & \Me{}, Lunney Borden, L., \& Wiseman, D. (2011, May).
  Can We Be Thankful for Mathematics? Mathematical Thinking and
  Aboriginal Peoples.  Proceedings of the 2010 Annual Meeting of the
  Canadian Mathematics Education Study Group = Actes de La Rencontre
  Annuelle 2010 Du Groupe Canadien d’Etude En Didactique Des
  Mathematiques (34th, Burnaby, British Columbia, Canada, May 21--25,
  2010).
  \Website{www.cmesg.org/wp-content/uploads/2015/01/CMESG2010.pdf}
  \\
  \Year{2006} & \Me{} (2007, May).  Mathematics as Medicine.
  Proceedings of the 2006 Annual Meeting of the Canadian Mathematics
  Education Study Group = Actes de La Rencontre Annuelle 2006 Du
  Groupe Canadien d’Etude En Didactique Des Mathematiques (30th,
  Calgary, Alberta, Canada, Jun 3--7, 2006).
  \Website{www.cmesg.org/wp-content/uploads/2015/01/CMESG2006.pdf}
\end{EntriesTableYear}

%%%%%%%%%%%%%%%%%%%%%%%%%%%%%%%%%%%%%%%%%%%%%%%%%%%%%%%%%%%%%%%%%%%%%%%%%%%%%%% 
\section{Peer Review and Editorial Work}

\begin{EntriesTableDuration}
  \Year{2025} & \textbf{Chair}, Expert Committee, Canada Foundation
  for Innovation (CFI) \Website{www.innovation.ca}
  \\
  \Duration{2024}{\Ongoing} & \textbf{Guest Editor}, \textit{Education
    Sciences} special issue Indigenous Pedagogies and Perspectives in
  STEM and Mathematics Education: Learning that Supports the Well-Being
  of Self, Family, Community, Land, and Ancestors
  \\
  \Year{2024} & \textbf{Referee} for the \textit{International Journal of
    Science and Mathematics Education}
  \\ %
  \Year{2022} & \textbf{Peer Review Member}, First Nations Biobanking and
  Genomic Research committee, Canadian Institutes of Health Research (CIHR)
  \Website{www.researchnet-recherchenet.ca/rnr16/vwOpprtntyDtls.do?prog=3635}
  \\ %
  \Year{2020} & \textbf{Referee} for the \textit{Engaged Scholar}
  journal: Community-Based Research, Teaching and Learning special
  issue on Trans-systemic and Indigenous Knowledges
  \\
  \Year{2018} & \textbf{Referee} for the \textit{Minnesota Journal of
    Undergraduate Mathematics}
  \Website{pubs.lib.umn.edu/index.php/mjum/}
  \\
  \Duration{2016}{2017} & \textbf{Referee} for \textit{Canadian Journal
    of Science, Mathematics and Technology Education} special issue on
  Indigenous Mathematics Education
  \\
  \Year{2014} & \textbf{Referee}, \textit{in education} journal
  \Website{journals.uregina.ca/ineducation}
  \\
  \Year{2012} & \textbf{Member}, Multidisciplinary Assessment
  Committee, Canada Foundation for Innovation (CFI)
  \newline
  \Website{www.innovation.ca}
  \\
  \Year{2012} & \textbf{Member}, Insight Grant Aboriginal
  Research Adjudication Committee, Social Sciences and Humanities Research
  Council (SSHRC) \Website{sshrc-crsh.gc.ca}
  \\
  \Year{2009} & \textbf{Referee} for the \textit{Canadian Journal of
    Science, Mathematics and Technology Education} special issue on
  Indigenous Science through Place
  \\
  \Duration{2005}{2009} & \textbf{Reviewer} for Pearson Education’s
  \textit{Math Makes Sense} K-12 textbook series for the Western and
  Northern Canadian Protocol (WNCP) curriculum
  \\
  \Duration{1996}{1997} & \textbf{Language Editor} for Ivrii,
  V. \textit{Microlocal Analysis and Precise Spectral Asymptotics}.
  Springer Verlag
\end{EntriesTableDuration}

%%%%%%%%%%%%%%%%%%%%%%%%%%%%%%%%%%%%%%%%%%%%%%%%%%%%%%%%%%%%%%%%%%%%%%%%%%%%%%%
\section{Reports}

\begin{EntriesTableYear}
  \Year{2000} & \Me{}. \textit{Report on the Creation of an Aboriginal
    Studies Course at Queen's University}
\end{EntriesTableYear}

%%%%%%%%%%%%%%%%%%%%%%%%%%%%%%%%%%%%%%%%%%%%%%%%%%%%%%%%%%%%%%%%%%%%%%%%%%%%%%%
\section{Problem Solutions}

\begin{EntriesTableYear}
  \Year{2006} & \Me{} (2006a).  Solution to Problem 3026.  Crux
  Mathematicorum, 32(3), 184--185. %
  \newline %
  \Me{} (2006b).  Solution to Problem 3028.  Crux Mathematicorum,
  32(3), 186--187. %
  \newline %
  \Me{} (2006c).  Solution to Problem 3029.  Crux Mathematicorum,
  32(3), 187--188. %
  \newline %
  \Website{cms.math.ca/publications/crux/issue/?volume=32\&issue=3}
  \\
  \Year{2006} & \Me{} (2006d).  Solution to Problem 3080.  Crux
  Mathematicorum, 32(7), 473--475. %
  \newline %
  \Website{cms.math.ca/publications/crux/issue/?volume=32\&issue=7}
  \\
  \Year{1988} & \Me{} (1988a). Solution to Problem 1985.305.8.  Crux
  Mathematicorum, 14(2), 42--43. %
  \newline %
  \Me{} (1988b).  Solution to Problem 1985.305.10.  Crux
  Mathematicorum, 14(2), 43--44. %
  \newline %
  \Website{cms.math.ca/publications/crux/issue/?volume=14\&issue=2}
  \\
  \Year{1988} & \Me{} (1988c).  Solution to Problem 1985.305.11.  Crux
  Mathematicorum, 14(3), 68--69.  \newline \Me{} (1988d).  Solution to
  Problem 1985.306.19.  Crux Mathematicorum, 14(3), 70--71.  \newline
  \Me{} (1988e).  Solution to Problem 1985.306.20.  Crux
  Mathematicorum, 14(3), 71.  \newline \Me{} (1988f).  Solution to
  Problem 1985.307.22.  Crux Mathematicorum, 14(3), 71--72.  \newline
  \Me{} (1988g).  Solution to Problem 1985.307.23.  Crux
  Mathematicorum, 14(3), 72.  \newline \Me{} (1988h).  Solution to
  Problem 1985.307.26.  Crux Mathematicorum, 14(3), 73–-74.  \newline
  \Me{} (1988i).  Solution to Problem 1985.307.27.  Crux
  Mathematicorum, 14(3), 74--75.
  \newline %
  \Website{cms.math.ca/publications/crux/issue/?volume=14\&issue=3}
  \\
  \Year{1988} & \Me{} (1988j).  Solution to Problem 1986.3.1.  Crux
  Mathematicorum, 14(4), 102--103.  \newline \Me{} (1988k).  Solution
  to Problem 1986.4.2.  Crux Mathematicorum, 14(4), 103.  \newline
  \Me{} (1988l).  Solution to Problem 1986.4.3.  Crux Mathematicorum,
  14(4), 103.  \newline \Me{} (1988m).  Solution to Problem 1986.4.4.
  Crux Mathematicorum, 14(4), 104.
  \newline %
  \Website{cms.math.ca/publications/crux/issue/?volume=14\&issue=4}
\end{EntriesTableYear}
      
%%%%%%%%%%%%%%%%%%%%%%%%%%%%%%%%%%%%%%%%%%%%%%%%%%%%%%%%%%%%%%%%%%%%%%%%%%%%%%%
\section{Open Software}

\begin{EntriesTableYear}
  \Year{2025} & \textbf{Curriculum Vitae}, adaptation of a GitHub
  template for developing a readable curriculum vitae in \LaTeX.
  \GitHub{edoolittle/cv}.
\end{EntriesTableYear}

%%%%%%%%%%%%%%%%%%%%%%%%%%%%%%%%%%%%%%%%%%%%%%%%%%%%%%%%%%%%%%%%%%%%%%%%%%%%%%% 
\section{Open Educational Resources}

\begin{EntriesTableYear}
  \Year{2025} & \textbf{Calculus I}, resources including Beamer
  slides, problem sets, and solutions, for use in conjunction with
  Stewart Calculus.  \GitHub{edoolittle/calculus-1}.
\end{EntriesTableYear}

%%%%%%%%%%%%%%%%%%%%%%%%%%%%%%%%%%%%%%%%%%%%%%%%%%%%%%%%%%%%%%%%%%%%%%%%%%%%%%% 
\section{Selected Presentations, Workshops, and Other Media}

\begin{EntriesTableYear}
  \Year{2025} & Keshen, J, Ottmann, J, Yost, C, \Me{}.  ReconciliAction
  panel, National Building Reconciliation Forum 2025, First Nations
  University of Canada and the University of Regina.
  \\
  \Year{2025} & \Me{}.  Division with Remainder: Indigenous Perspectives.
  Alberta Mathematics Dialogue 2025, University of Calgary.
  \Website{tinyurl.com/division-with-remainder-pptx}
  \\
  \Year{2025} & \Me{}.  Tips and Tricks for UR Courses.  FNUniv Instructors
  Lunch and Learn, First Nations University of Canada.
  \Website{urcourses.uregina.ca/mod/page/view.php?id=2932292\&forceview=1}
  \\
  \Year{2025} & \Me{}.  Mathematics as Story.  Storytelling through
  Art, Language, and Action.  State University of New York at Buffalo.
  % Website is not great ... is there more detail somewhere?
  % https://www.ub-connect.com/s/1703/alumni/index.aspx?sid=1703&pgid=5146&gid=2&cid=8945&ecid=8945&post_id=0
  % Try this
  %\Website{arts-sciences.buffalo.edu/indigenous-studies/news-events/storytellers-conference.html}
  \\
  \Year{2024} & \Me{}.  String Figures and Knots.  
  Aboriginal and Torres Strait Islander Mathematics Alliance
  STEM Camp, Birrigai Outdoor School, Australian Capital Territory.
  %\Website{https://atsima.com/stem-steam-camps/}
  \\
  % TODO: Talk at Newcastle
  % TODO: Talk at University of Queensland
  \Year{2024} & \Me{}.  Mathematics as a Spiritual Being.  Invited
  talk at the Department of Mathematics, University of Saskatchewan
  \Website{usask.cloud.panopto.eu/Panopto/Pages/Viewer.aspx?id=001c1142-f3a6-476b-bed9-b13c0037c04f}
  \\
  \Year{2024} & \Me{}.  Presenter and Organizer for Indigenising
  University Mathematics 3 international conference held at La Trobe
  University, Melbourne, Australia
  \\
  \Year{2023} & \Me{}.  Better Living Through Combinatorics.  Topic
  session at the 2023 Canadian Math Education Study Group Conference.
  \\
  \Year{2023} & \Me{}.  (2023, September 1).  Indigenous Math Podcast
  1 (No. 1) [Broadcast].  \Website{www.cfnuradio.ca/voices/}
  \\
  \Year{2023} & \Me{}, \& Czuy, K. (2023, June 3).  Mathematics is
  Creation, Being, \& Medicine (No. 15) [Broadcast].
  \Website{open.spotify.com/episode/21oDjAQvZIOVWIIRh3CaKS}
  \\
  \Year{2023} & \Me{}. Indigenizing University Mathematics. Invited plenary
  speaker at Alberta Mathematics Dialogue.
  %\Website{sites.google.com/mtroyal.ca/amd/plenary-speakers?authuser=0}
  %\\
  %\Year{2022} & \Me{}. Indigenous Maths, Global Math, and Indigenizing
  %Mathematics.  Faculty of Mathematics, University of Waterloo.
  %\Website{uwaterloo.ca/math/events/guest-lecture-indigenization-mathematics-dredward-doolittle}
  \\
  \Year{2022} & \Me{}. Indigenous Maths, Global Math, and Indigenizing
  Mathematics.  The Centre for Indigenous Knowledges and Languages, and the
  Department of Mathematics and Statistics, York University.
  \Website{www.youtube.com/watch?v=ptk\_\_Ga43Wg}
  \\
  \Year{2022} & \Me{}. Presenter and Organizer for Indigenizing
  University Mathematics 2 international conference held at the
  University of Newcastle, Australia, and at First Nations University
  of Canada
  \\
  \Year{2022} & \Me{}. ‘Tent Talk’ on Indigenous Mathematics.  Teaching
  and Learning Here and Now conference at the University of Regina
  \\
  \Year{2022} & \Me{}, \& Native Stories.  (2022, April 3).  Native
  Stories: Indigenous Mathematicians: Edward Doolittle [Broadcast].
  \Website{nativestories.org/indigenous-mathematicians-edward-doolittle/}
  \\
  \Year{2021} & Borwein, N., and \Me{}. Indigenous Mathematics.
  Keynote speakers at the Indigenising University Mathematics 2021
  conference, University of Newcastle, Australia
  \newline
  \Website{carmamaths.org/meetings/ium/video/Theme 4 - Edward and Naomi.mp4}
  \\
  \Year{2020} & \Me{}. Online presentation on Indigenous mathematics
  for Teachers for the Federation of Sovereign Indigenous Nations
  (FSIN), Saskatchewan
  \\
  \Year{2020} & \Me{}. Online presentation for Langara College on
  Indigenizing math education at the post-secondary level
  \\
  \Year{2019} & \Me{}, \& Russell, G.  Mathematics as a Tool for
  Colonization, keynote presentation at Provoking Curriculum,
  Faculty of Education, University of Regina
  %\Website{viewer.joomag.com/education-news-spring-2019/0753087001637684188/p15}
  \\
  \Year{2019} & \Me{}, \& Nolan, K.  Initiating and Nurturing
  Collaborations Between Mathematicians and Mathematics Educators,
  panel presentation at the Canadian Mathematics Education Study Group
  annual meeting
  \\
  \Year{2019} & \Me{}.  Word Puzzles in Indigenous Languages,
  presentation at the Canadian Mathematical Society summer meeting
  \\
  \Year{2019} & \Me{}.  Mathematics of Indigenous Games, a talk for high
  school students at Campbell Collegiate high school in Regina
  \\
  \Year{2019} & \Me{}.  MATH 101 for Indigenous Students, presentation
  at the Canadian Mathematical Society winter meeting %
  \\ %
  \Year{2019} & Leung, F.-S., \Me{}, Zazkis, R., \& Marken, K.
  Looking In, panel presentation at Innovations in New Instructor
  Training, Banff International Research Station workshop 19w2231
  \newline %
  \Website{www.birs.ca/events/2019/2-day-workshops/19w2231/videos/watch/201906220942-Doolittle.html}
  \\
  \Year{2018} & \Me{}.  Off the Grid, keynote presentation at Living
  Mathematics in Our Communities, the 8th Aboriginal Mathematics
  Symposium at the First Nations Longhouse, University of British
  Columbia
  \\
  \Year{2018} & Bazzul, J., \Me{}, Omeasoo, D., \& Russell, G.
  Exploring Indigenous Spiritualities In-Relation: How Might Science
  and Math Education Become Different?  Panel at the Canadian Society
  for the Study of Education conference
  %\Website{csse-scee.ca/wp-content/uploads/2018/03/2018_CSSE_PC_Prog_CSSE_20180327_Complete.pdf}
  \\
  \Year{2018} & \Me{}.  What is Indigenous Mathematics?  Weweni Lecture,
  University of Winnipeg.  \newline
  \Website{www.uwinnipeg.ca/indigenous/weweni/weweni-2018/what-is-indigenous-mathematics.html}
  \\
  \Year{2017} & \Me{}.  Harmonics, Nodal Lines, and Acoustic Levitation,
  keynote presentation at the Treaty 4 Math Fair in Fort Qu’appelle
  \\
  \Year{2017} & \Me{}.  Transformations, Symmetry, and the Starblanket,
  a talk for Grade 9 students at Campbell Collegiate high school in
  Regina
  \\
  \Year{2016} & \Me{}.  From String Figures to the Fields Medal at the
  6th Aboriginal Students in Math and Science Workshop for Indigenous
  high school students at Simon Fraser University
  \\
  \Year{2016} & \Me{}.  An Exploration of the Fibonacci Sequence at
  Discover Your Direction, an event to encourage Indigenous grade 10
  students to attend university
  \\
  \Year{2016} & \Me{}.  Indigenous String Figures for the Science
  Showcase Series, First Nations University
  \\
  \Year{2015} & \Me{}.  The Development of a Plains Cree Pangrammatic
  Autogram at STEMfest, an international conference, in Saskatoon
  \\
  \Year{2015} & \Me{}.  Indigenous Students of Mathematics, invited talk
  at the National Meeting of Canadian Math Department Chairs
  \\
  \Year{2014} & \Me{}.  Cree Syllabic Crosswords, FNUniv Endangered
  Alphabets conference
  \\
  \Year{2014} & \Me{}.  Indigenous Math Education, University of
  Regina/High School Transitions Committee Joint Professional
  Development conference
  \\
  \Year{2014} & \Me{}.  Native American Mathematics, Richard and Louise
  Guy Lecture, University of Calgary \newline
  \Website{mathtube.org/lecture/video/native-american-mathematics}
  \\
  \Year{2014} & \Me{}.  Indigenous Mathematics, Invited keynote address
  at the Ontario Institute for Studies in Education conference N’gwii
  Kendaasmin (We’ll Learn Together): Drawing on Indigenous Knowledges
  to Transform Teaching and Learning in Mathematics and Science
  \\
  \Year{2014} & \Me{}.  Indigenous Mathematics, Invited talk for
  Indigenous Math and Science conference at the University of Manitoba
  \\
  \Year{2013} & \Me{}.  Mathematics of Planet Earth: Graph Theory of the
  Food Chain for the Science Camp for Aboriginal Youth, First Nations
  University
  \\
  \Year{2013} & \Me{}.  Word Puzzles in Cree at the fourth annual Math
  and Science for Aboriginal Students conference at Simon Fraser
  University
  \\
  \Year{2013} & \Me{}.  Manipulatives in High School Math Education
  series of workshops for the Yorkton Tribal Council %
  \\ %
  \Year{2012} & \Me{}.  Graph Theory in an Indigenous Context, in First
  Nations Math Education, Banff International Research Station
  workshop 12w5076 %
  \newline %
  \Website{www.birs.ca/events/2012/5-day-workshops/12w5076/videos/watch/201211210914-Doolittle.html} %
  \\ %
  \Year{2012} & \Me{}.  Seeing Sounds, Science Camp for
  Aboriginal Youth, First Nations University
  \\
  \Year{2011} & \Me{}.  Mazes and Explorobots for the Science Camp for
  Aboriginal Youth, First Nations University
  \\
  \Year{2010} & \Me{}.  Aboriginal Perspectives in the Saskatchewan
  Mathematics Curriculum, Emerging Professionalism Conference,
  Faculty of Education, University of Regina
  \\
  \Year{2009} & \Me{}.  Music and Signal Processing, presentation to
  Aboriginal K-12 students at the Kehewin Education Institute
  \\
  \Year{2009} & \Me{}.  Teaching Aboriginal Perspectives in High School
  Mathematics at the Sun Country School District Professional
  Development Conference
  \\
  \Year{2005} & \Me{}.  Building Community through Science Curriculum
  Actualization, Awasis Conference (organized by the Saskatchewan
  Teachers’ Federation)
  \\
  \Year{2002} & \Me{}.  Knots in Aboriginal Mathematics Education, Guest
  lecture for the Aboriginal Teacher Education Program, Queen’s
  University
  \\
  \Year{1998} & \Me{}.  Issues in the Mathematics Education of
  Aboriginal Students, Engineering Explorations Concordia University
  \\
  \Year{1998} & \Me{}.  Seeing Sounds, Blueprint for the Future,
  National Aboriginal Achievement Foundation
  \\
  \Year{1998} & \Me{}.  Graduation Address, Grand River Post-Secondary
  Education student recognition dinner
  \\
  \Year{1997} & \Me{}.  Experiences of Indigenous Students, Native
  Science Dialogue, University of Toronto
  \\
  \Year{1997} & \Me{}.  Mathematics as a Spiritual Endeavour, Aboriginal
  Youth Career Symposium
  \\
  \Year{1996} & \Me{}.  Introduction of Hopi filmmaker Victor Masayesva
  Jr., Editing Aboriginal Oral Texts: the thirty-second annual Conference
  on Editorial Problems, University of Toronto
  \Website{coilink.org/20.500.12592/xdq1t6}
  \\
  \Year{1995} & \Me{}.  Native Mathematics, Native Issues Seminar, First
  Nations House, University of Toronto
  \\
  \Year{1995} & \Me{}.  Native Mathematics Education, Native Science
  Teachers Camp, University of Toronto
\end{EntriesTableYear}

%%%%%%%%%%%%%%%%%%%%%%%%%%%%%%%%%%%%%%%%%%%%%%%%%%%%%%%%%%%%%%%%%%%%%%%%%%%%%%%
\section{Selected Teaching Experience}

\begin{EntriesTableDuration}
  \Duration{2020}{\Ongoing} & \textbf{Professor}, MATH 110 (Calculus
  I) remote modality (online, synchronous), First Nations University
  of Canada
  \\
  \Duration{2020}{\Ongoing} & \textbf{Mentor}, Putnam Competition
  Training, University of Regina
  % Putnam Training leader for the University of Regina’s Putnam
  % Competition team.  The 2020 competition (written in February 2021,
  % due to COVID) is the first Putnam competition in which the
  % University of Regina participated.  We have since participated in
  % four more Putnam competitions
  %
  % 2001-present A wide cross-section of University of Regina and
  % First Nations University undergraduate mathematics and statistics
  % courses including calculus, linear algebra, differential
  % equations, finite mathematics, adult mathematics, the history of
  % mathematics, and statistics
  \\
  \Year{2023} & \textbf{Sessional Lecturer}, EDPJ/EDJI 1100 (Mathematics
  Education for Primary/Junior/Intermediate) for the Waaban Indigenous
  Education Program at York University
  \\
  \Duration{2013}{2014} & \textbf{Professor}, AMTH 001/091/092 (Adult
  Mathematics) for community-based access programs in Onion Lake and
  Piapot First Nations, First Nations University
  \\
  \Year{2012} & \textbf{Professor}, MATH 101 for the community-based
  Aboriginal Teacher Education Program (Fort Qu’appelle), First
  Nations University
  \\
  % 2009,
  \Year{2018} & \textbf{Professor}, EMTH 215 (Elementary Mathematics
  Education), Department of Indigenous Education, First Nations
  University
  \\
  \Year{2001} & \textbf{Instructor}, Leadership and Management, Grand
  River Polytechnic, Six Nations
  \\
  \Year{2000} & \textbf{Lecturer}, Native Studies, Faculty of
  Environmental Studies, York University
  \\
  \Duration{1999}{2001} & \textbf{Lecturer}, ABS201Y (Aboriginal
  Studies), Faculty of Arts and Science, University of Toronto
  % Responsibilities included course design and delivery for the first
  % and second offerings of ABS201, Introduction to Aboriginal
  % Studies, at the University of Toronto; participation in the
  % selection and supervision of a teaching assistant; and lecturing
  % weekly to a class of 90 students.
  \\
  \Duration{1998}{1999} & \textbf{Instructor}, Elementary Mathematics
  Education, Aboriginal Teacher Education Program, Faculty of
  Education, Queen's University
  % My duties in ATEP included traveling to remote Northern Ontario
  % communities to deliver program material to Ontario Teacher's
  % Certificate candidates, developing curriculum for Aboriginal
  % students and teachers, and grading papers.
  \\
  \Year{1999} & \textbf{Instructor}, Business Math, First Nation
  Management Training and Confederation College, Sioux Lookout
  % In the Confederation College program, I taught a second year
  % college-level course in business mathematics to northern Ontario
  % Native band administrators in the town of Sioux Lookout.  I was
  % responsible for selecting textbooks, developing course outline,
  % producing and marking tests and a final exam, supervising one
  % teaching assistant, and delivering a report on the experience.
  \\
  \Duration{1997}{1998} & \textbf{Lecturer}, MAT 188F/196F (Linear
  Algebra/Calculus), Faculty of Engineering, University of Toronto
  % Duties included lecturing first-year university calculus and
  % linear algebra three times weekly to a class of 80-100 students,
  % testing the students during the term, and supervising two teaching
  % assistants.
  \\
  \Year{1997} & \textbf{Lecturer}, MAT135Y (Calculus I), Faculty of
  Arts and Science, University of Toronto
  % Duties included course coordination, planning lessons, tests and
  % problem sets, scheduling tests, lecturing four times weekly, and
  % supervising three teaching assistants.
  \\
  \Year{1996} & \textbf{Lecturer}, MAT186F (Calculus IB), Faculty of
  Engineering, University of Toronto
  % Duties included lecturing in first-year university calculus three
  % times weekly to a class of 126 students, testing the students during
  % the term, and supervising two teaching assistants.
  % 
  % 1996 Instructor and Mentor, Summer Opportunity in Applied Research,
  % University of Toronto
  % 
  % I taught computing concepts and fractal and dynamical systems
  % programming in C++ and Maple to advanced high school students;
  % devised curriculum; was responsible for computer support on a
  % network of 30 computers; authored for the World Wide Web.
  % 
  % 1996 Instructor, computer course for Aboriginal Legal Services
  % Paralegal Training
  % 
  % I taught elementary computing concepts and basic WordPerfect and
  % Lotus techniques to a group of Native paralegal students; attempted
  % to relate computing concepts to traditional Native learning
  % paradigms.
  % 
  \\
  \Duration{1992}{1994} & \textbf{Academic Counselor}, First Nations
  House, University of Toronto
  \\
  \Duration{1986}{1996} & \textbf{Teaching Assistant}, Department of
  Mathematics, University of Toronto
  % Courses included Complex Analysis, Differential Equations for
  % Engineering Science, Calculus for Life Sciences I and II, Analysis
  % I for Mathematics Specialists, Linear Algebra I for Mathematics
  % Specialists and Calculus for Finance
\end{EntriesTableDuration}


%%%%%%%%%%%%%%%%%%%%%%%%%%%%%%%%%%%%%%%%%%%%%%%%%%%%%%%%%%%%%%%%%%%%%%%%%%%%%%%
\section{Selected Community Service}

\begin{EntriesTableDuration}
  \Duration{2023}{2025} & \textbf{Member}, Academic Performance Review
  Committee, First Nations University of Canada
  \Website{www.fnuniv.ca}
  \\
  \Duration{2021}{\Ongoing} & \textbf{Member}, Canada Jay Mathematical
  Competition committee, Canadian Mathematical Society
  \newline
  \Website{cms.math.ca/competitions/cjmc/}
  \\
  \Duration{2019}{\Ongoing} & \textbf{Member}, Bargaining Team, First
  Nations University of Canada Academic Bargaining Unit, University of
  Regina Faculty Association \Website{www.urfa.ca}
  \\
  \Duration{2015}{\Ongoing} & \textbf{Member}, Pension and Benefits
  Committee, First Nations University of Canada
  \Website{www.fnuniv.ca}
  \\
  \Year{2018} & \textbf{Judge}, Yakutia International Science Fair in
  Yakutsk, Russia \Website{ysf.lensky-kray.ru/en/}
  \\
  \Duration{2018}{2022} & \textbf{Mentor}, Verna J. Kirkness
  Foundation program \Website{www.vernajkirkness.org}
  \\
  \Duration{2017}{\Ongoing} & \textbf{Judge}, and \textbf{Chair of a
    Judging Group}, Canada Wide Science Fair \Website{cwsf-espc.ca}
  \\
  \Year{2017} & \textbf{Judge}, File Hills Qu'appelle Tribal Council
  Science Fair
  \\
  \Year{2017} & \textbf{Judge}, Treaty 4 Math Fair
  % \\
  % 2016-2018 Member of Revisioning, Reclaiming, Reconciling School
  % Mathematics, a group of academics, educators, and administrators
  % developing Indigenous math curriculum and policy proposals for the
  % next round of K-12 math curriculum revision in Saskatchewan.
  \\
  \Duration{2015}{2017} & \textbf{Secretary/Treasurer}, Native
  Heritage Foundation of Canada
  \\
  \Duration{2013}{2014} & \textbf{Chair}, First Nations Environmental
  Contaminants Program (FNECP) Selection Committee %
  \newline %
  \Website{www.sac-isc.gc.ca/eng/1583779185601/1583779243216}
  \\
  \Year{2010} & \textbf{Tax Policy Consultant}, Chiefs of Ontario
  \Website{chiefs-of-ontario.org}
  \\
  \Year{2010} & \textbf{Member}, Academic Reform Task Force, First
  Nations University of Canada \Website{www.fnuniv.ca}
  \\
  \Year{2010} & \textbf{Webmaster}, \textit{Fund First Nations
    University Now!} blog \Website{fnuniv.wordpress.com}
  \\
  \Duration{2007}{2008} & \textbf{Member}, Canadian Math Olympiad
  committee \Website{cms.math.ca/competitions/cmo/}
  \\
  \Year{2007} & \textbf{Mentor}, Canada’s International Math Olympiad
  team \Website{cms.math.ca/competitions/imo/}
  \\
  \Duration{2003}{2005} & \textbf{Member}, President’s International
  Alumni Council, University of Toronto \Website{www.utoronto.ca}
  %
  % 2001 Named one of 50 outstanding alumni of the University of
  % Toronto in the University of Toronto Magazine 28(4) (Summer 2001)
  \\
  \Duration{1999}{2002} & \textbf{Member}, National Aboriginal
  Achievement Foundation Postsecondary Awards Jury
  %
  % 1997 Named one of the “Great Minds at the University of Toronto”
  % as part of their $300M fundraising campaign
  \\
  \Duration{1996}{1997} & \textbf{Graduate Student Representative},
  Aboriginal Advisory Council, University of Toronto
  \Website{www.utoronto.ca}
  %
  % 1994-1997 Participant in the Native Issues Seminar for Graduate
  % Students and Faculty, First Nations House, University of Toronto
  \\
  \Duration{1991}{1992} & \textbf{Member}, Presidential Advisory
  Committee on Race Relations and Anti-Racism Initiatives, University
  of Toronto \Website{www.utoronto.ca}
  %
  % 1991-1992 Advisory Committee on the Appointment of the Warden of
  % Hart House
  %
  % 1990-1991 Presidential Review of Hart House, University of Toronto
  %
  % 1989-1990 Vice President, Native Students' Association, University
  % of Toronto
  %
  % 1988-1989 Executive Member, Canadian Union of Educational Workers
  % Local 2
  %
  % 1987-1988 President, Mathematics and Statistics Student Union,
  % University of Toronto
\end{EntriesTableDuration}

%% From original leouieda/cv

% %%%%%%%%%%%%%%%%%%%%%%%%%%%%%%%%%%%%%%%%%%%%%%%%%%%%%%%%%%%%%%%%%%%%%%%%%%%%%%%
% \section{Open Research Software}

% \begin{EntriesTableDuration}
%   \Duration{2010}{\Ongoing} &
%   \textbf{Fatiando a Terra} | \Website{www.fatiando.org}
%   \newline
%   \textit{Python tools for geophysical data processing, forward modeling, and inversion}
%   \newline
%   Role: Project founder, core developer, Steering Council Member
%   \\
%   \Duration{2017}{\Ongoing} &
%   \textbf{The Generic Mapping Tools (GMT)} | \Website{www.generic-mapping-tools.org}
%   \newline
%   \textit{A data processing and mapping toolbox for the Earth, Ocean, and Planetary Science}
%   \newline
%   Role: Community stewardship advisor, set up the website + forum + GitHub workflow
%   \\
%   \Duration{2022}{\Ongoing} &
%   \textbf{xlandsat} | \Website{www.compgeolab.org/xlandsat}
%   \newline
%   \textit{Load Landsat remote sensing scenes in Python and xarray}
%   \newline
%   Role: Creator and sole developer
%   \\
%   \Duration{2017}{2021} &
%   \textbf{PyGMT} | \Website{www.pygmt.org}
%   \newline
%   \textit{A Python interface for the Generic Mapping Tools}
%   \newline
%   Role: Project founder, developer, advisor
%   \\
%   \Duration{2009}{2016} &
%   \textbf{Tesseroids} | \Website{tesseroids.leouieda.com}
%   \newline
%   \textit{Forward modeling of gravitational fields in spherical coordinates}
%   \newline
%   Role: Creator and sole developer
% \end{EntriesTableDuration}


% %%%%%%%%%%%%%%%%%%%%%%%%%%%%%%%%%%%%%%%%%%%%%%%%%%%%%%%%%%%%%%%%%%%%%%%%%%%%%%%
% \section{Selected Invited Presentations}

% \begin{EntriesTableYear}
% \Year{2021}  &
%   \textbf{Design useful tools that do one thing well and work together: rediscovering the UNIX philosophy while building the Fatiando a Terra project}.
%   \newline
%   AGU 2021.
%   \Me; \LLi; \Santiago; \Agustina.
%   \GitHub{fatiando/agu2021}.
%   \\
%   &
%   \textbf{Open-science for gravimetry: tools, challenges, and opportunities}.
%   \newline
%   GFZ Helmholtz Centre Potsdam.
%   \Me; \Santiago; \Agustina.
%   \GitHub{leouieda/2021-06-22-gfz}.
%   \\
%   &
%   \textbf{Fatiando a Terra: Open-source tools for geophysics}.
%   \newline
%   Geophysical Society of Houston.
%   \Me; \Santiago; \Agustina.
%   \GitHub{fatiando/2021-gsh}.
%   \\
% \Year{2020}  &
%   \textbf{Geophysical research powered by open-source}.
%   \newline
%   Christian Albrechts Universität zu Kiel.
%   \Me.
%   \GitHub{leouieda/2020-07-01-kiel}.
% \end{EntriesTableYear}

% %%%%%%%%%%%%%%%%%%%%%%%%%%%%%%%%%%%%%%%%%%%%%%%%%%%%%%%%%%%%%%%%%%%%%%%%%%%%%%%
% \section{Publication Highlights}

% \begin{EntriesTableYear}
% \Year{2025}  &
%   \textbf{Euler inversion: Locating sources of potential-field data through inversion of Euler's homogeneity equation}.
%   \newline
%   \Me; \Gelson; \India; \Bi.
%   EarthArXiv.
%   \DOI{10.31223/X5T41M}
%   \newline
%   Open science:
%   \GitHub{compgeolab/euler-inversion}
%   |
%   \Data{10.6084/m9.figshare.26384140}
%   \\
% \Year{2024}  &
%   \textbf{Full vector inversion of magnetic microscopy images using Euler deconvolution as prior information}.
%   \newline
%   \Gelson; \Me; \emph{et al}.
%   Geochemistry, Geophysics, Geosystems.
%   \DOI{10.1029/2023GC011082}
%   \newline
%   Open science:
%   \GitHub{compgeolab/micromag-euler-dipole}
%   |
%   \Data{10.6084/m9.figshare.22672978}
%   \\
% \Year{2021}  &
%   \textbf{Gradient-boosted equivalent sources}.
%   \newline
%   \Santiago; \Me.
%   Geophysical Journal International.
%   \DOI{10.1093/gji/ggab297}
%   \newline
%   Open science:
%   \GitHub{compgeolab/eql-gradient-boosted}
%   |
%   \Data{10.6084/m9.figshare.13604360}
%   \\
% \Year{2020}  &
%   \textbf{Pooch: A friend to fetch your data files}.
%   \newline
%   \Me; \Santiago; \Remi; \Hugo; \emph{et al}.
%   Journal of Open Source Software.
%   \DOI{10.21105/joss.01943}
%   \newline
%   Open science:
%   \GitHub{fatiando/pooch}
%   |
%   \Data{10.5281/zenodo.3515030}
%   \\
% \Year{2019}  &
%   \textbf{The Generic Mapping Tools, Version 6}.
%   \newline
%   \Paul; \Joaquim; \Me; \emph{et al}.
%   Geochemistry, Geophysics, Geosystems.
%   \DOI{10.1029/2019GC008515}
%   \newline
%   Open science:
%   \GitHub{GenericMappingTools/gmt}
%   \\
%   \Year{2019}  &
%   \textbf{Gravitational field calculation in spherical coordinates using variable densities in depth}.
%   \newline
%   \Santiago; \Agustina; \Gimenez; \Me.
%   Geophysical Journal International.
%   \DOI{10.1093/gji/ggz277}
%   \newline
%   Open science:
%   \GitHub{pinga-lab/tesseroid-variable-density}
%   |
%   \Data{10.6084/m9.figshare.8239622}
%   \\
% \Year{2018}  &
%   \textbf{Verde: Processing and gridding spatial data using Green's functions}.
%   \newline
%   \Me.
%   Journal of Open Source Software.
%   \DOI{10.21105/joss.00957}
%   \newline
%   Open science:
%   \GitHub{fatiando/verde}
%   |
%   \Data{10.5281/zenodo.1478244}
%   \\
% \Year{2017}  &
%   \textbf{Fast non-linear gravity inversion in spherical coordinates with application to the South American Moho}.
%   \newline
%   \Me; \Val.
%   Geophysical Journal International.
%   \DOI{10.1093/gji/ggw390}
%   \newline
%   Open science:
%   \GitHub{pinga-lab/paper-moho-inversion-tesseroids}
%   |
%   \Data{10.6084/m9.figshare.3987267}
%   \\
% \Year{2016}  &
%   \textbf{Tesseroids: forward modeling gravitational fields in spherical coordinates}.
%   \newline
%   \Me; \Val; \Carla.
%   Geophysics.
%   \DOI{10.1190/geo2015-0204.1}
%   \newline
%   Open science:
%   \GitHub{pinga-lab/paper-tesseroids}
%   |
%   \Data{10.6084/m9.figshare.786514}
% \end{EntriesTableYear}

\end{document}
